%\documentclass[11pt]{article}
\documentclass[preprintnumbers,prd,superscriptaddress,notitlepage,nofootinbib] {revtex4-1}

\usepackage{geometry, amsmath, amsthm, latexsym, amssymb, graphicx}
\geometry{margin=1in, headsep=0.25in}
\usepackage[usenames, dvipsnames]{color}

% define units, short-names...

\newcommand{\frachalf}{\frac{1}{2}}
\newcommand{\dzbar}{\delta_{\bar{z}}}
\newcommand{\fNL}{f_{\rm NL}}
\newcommand{\kNL}{k_{\rm NL}}
\newcommand{\FoM}{{\rm FoM}}
\newcommand{\cm}{\ \text{cm}}
\newcommand{\pc}{\ \text{pc}}
\newcommand{\Mpc}{\ \text{Mpc}}
\newcommand{\Gpc}{\ \text{Gpc}}
\newcommand{\kpc}{\ \text{kpc}}
\newcommand{\Gyr}{\ \text{Gyr}}
\newcommand{\hkpc}{\ h^{-1}\text{kpc}}
\newcommand{\hMpc}{\ h^{-1}\text{Mpc}}
\newcommand{\hMpcc}{\ h^{-3}\text{Mpc}^3}
\newcommand{\hGpc}{\ h^{-1}\text{Gpc}}
\newcommand{\hGpcc}{\ h^{-3}\text{Gpc}^3}
\newcommand{\ihMpc}{\ h\text{Mpc}^{-1}}
\newcommand{\iMpc}{\ \text{Mpc}^{-1}}
\newcommand{\Ms}{\ M_\odot}
\newcommand{\hMs}{\ h^{-1} M_\odot}
\newcommand{\eh}[1]{\exp{\left[#1\right]}}
\newcommand{\tim}[1]{\times 10^{#1}}%times 10 
\newcommand{\unit}[1]{\ \text{#1}}
\newcommand{\tb}[1]{\textcolor{blue}{#1}}
\newcommand{\tr}[1]{\textcolor{red}{#1}}
\newcommand{\tsc}[1]{#1}
\newcommand{\derd}{\,\mathrm{d}} % upright d for derivatives and integrals
\newcommand{\ddir}{\delta^\text{(D)}}
\newcommand{\dkron}{\delta^\text{(K)}}
\newcommand{\be}{\begin{equation}}
\newcommand{\ee}{\end{equation}}
\newcommand{\la}{\left\langle}
\newcommand{\ra}{\right\rangle}
\newcommand{\derivd}{\text{d}}
\renewcommand{\vec}{\bm}
\newcommand{\FT}[1]{{\rm FT}\left[#1\right]}
\newcommand{\twopicub}{\left(2\pi\right)^3}
\newcommand{\intvkop}{\int\frac{\derivd^3k}{\twopicub}}
\newcommand{\intvx}{\int \derivd^3x~}
\newcommand{\dqc}{\frac{\derivd^3q}{(2\pi)^3}}
\newcommand{\intvqop}{\int \dqc}
\newcommand{\dkpc}{\frac{\derivd^3k'}{(2\pi)^3}}
\newcommand{\dqcp}{\frac{\derivd^3q'}{(2\pi)^3}}
\newcommand{\dqcpp}{\frac{\derivd^3q''}{(2\pi)^3}}
\newcommand{\cyc}{\, \text{cyc.}}
\newcommand{\Dz}{\Delta z}
\newcommand{\Dv}{\Delta v}
\newcommand{\Dx}{\Delta x}
\newcommand{\Deta}{\Delta \eta}
\newcommand{\Dtheta}{\Delta \theta}
\newcommand{\bz}{\bar{z}}
\newcommand{\kms}{{\rm km~s^{-1}}}
\newcommand{\ikms}{{\rm s~km^{-1}}}
\newcommand{\kmsMpc}{\nicefrac[\rm]{km}{s\,Mpc}}
\newcommand{\summnu}{\Sigma m_\nu}
\newcommand{\Nnueff}{N_{\nu,{\rm eff}}}
\newcommand{\kmaxeff}{k_{\rm max,eff}}
\newcommand{\kmax}{{k_{\rm max}}}
\newcommand{\kmin}{{k_{\rm min}}}
\newcommand{\kNyq}{{k_{\rm Nyq}}}
\newcommand{\neff}{{n_{\rm eff}}}
\newcommand{\sL}{\mathcal{L}}
\newcommand{\sN}{\mathcal{N}}
\newcommand{\sH}{\mathcal{H}}
\newcommand{\sD}{\mathcal{D}}
\newcommand{\gmunu}{{g_{\mu \nu}}}
\newcommand{\keV}{{\rm keV}}
\newcommand{\GeV}{{\rm GeV}}
\newcommand{\erg}{\, {\rm erg}}
\newcommand{\gram}{\, {\rm g}}
\newcommand{\kelvin}{\, {\rm K}}
\newcommand{\deltahat}{\hat{\delta}}

%bias parameters
\newcommand{\bdelta}{b_\delta}
\newcommand{\bdtwo}{b_{\delta^2}}
\newcommand{\bstwo}{b_{s^2}}
%renormalized fields
\newcommand{\msdtwo}{\left[\delta^2\right]}
\newcommand{\msstwo}{\left[s^2\right]}
%bold faced vectors
\newcommand{\vtheta}{\mathbf{\theta}}
\newcommand{\vDtheta}{\mathbf{\Delta \theta}}
\newcommand{\vv}{\mathbf{v}}
\newcommand{\vnabla}{\mathbf{\nabla}}
\newcommand{\veta}{\mathbf{\eta}}
\newcommand{\vk}{\mathbf{k}}
\newcommand{\vq}{\mathbf{q}}
\newcommand{\vt}{\mathbf{t}}
\newcommand{\vb}{\mathbf{b}}
\newcommand{\vx}{\mathbf{x}}
\newcommand{\vc}{\mathbf{c}}
\newcommand{\vone}{\mathbf{1}}
\newcommand{\vmu}{\mathbf{\mu}}
\newcommand{\vzeta}{\mathbf{\zeta}}
\newcommand{\vn}{\mathbf{n}}
\newcommand{\vgamma}{\mathbf{\gamma}}
\newcommand{\vpi}{\mathbf{\pi}}
\newcommand{\vrho}{\mathbf{\rho}}
\newcommand{\vphi}{\mathbf{\phi}}
\newcommand{\vchi}{\mathbf{\chi}}
\newcommand{\vomega}{\mathbf{\omega}}
\newcommand{\vu}{\mathbf{u}}
\newcommand{\vj}{\mathbf{j}}
\newcommand{\vg}{\mathbf{g}}
\newcommand{\vm}{\mathbf{m}}
\newcommand{\vd}{\mathbf{d}}
\newcommand{\vdelta}{\mathbf{\delta}}
\newcommand{\vy}{\mathbf{y}}
\newcommand{\vf}{\mathbf{f}}
\newcommand{\vp}{\mathbf{p}}
\newcommand{\vep}{\mathbf{\epsilon}}
\newcommand{\vo}{\mathbf{o}}
\newcommand{\vs}{\mathbf{s}}
\newcommand{\vsh}{\hat{\mathbf{s}}}
\newcommand{\vS}{\mathbf{S}}
\newcommand{\vC}{\mathbf{C}}
\newcommand{\vXi}{\mathbf{\Xi}}
\newcommand{\vQ}{\mathbf{Q}}
\newcommand{\vDelta}{\mathbf{\Delta}}
\newcommand{\vP}{\mathbf{P}}
\newcommand{\vN}{\mathbf{N}}
\newcommand{\vA}{\mathbf{A}}
\newcommand{\vM}{\mathbf{M}}
\newcommand{\vK}{\mathbf{K}}
\newcommand{\vB}{\mathbf{B}}
\newcommand{\vU}{\mathbf{U}}
\newcommand{\vT}{\mathbf{T}}
\newcommand{\vF}{\mathbf{F}}
\newcommand{\br}{\mathbf{r}}
\newcommand{\vR}{\mathbf{R}}
\newcommand{\Tr}{\mathrm{Tr}}
\newcommand{\lnL}{{\mathcal{L}}}
\newcommand{\vxperp}{\mathbf{x_\perp}}
\newcommand{\vkperp}{\mathbf{k_\perp}}
\newcommand{\kperp}{k_\perp}
\newcommand{\vrperp}{\mathbf{r_\perp}}
\newcommand{\vpar}{v_\parallel}
\newcommand{\rpar}{r_\parallel}
\newcommand{\kpar}{k_\parallel}
\newcommand{\tk}{\tilde{k}}
\newcommand{\tmu}{\tilde{\mu}}
\newcommand{\PtD}{P_{\rm 3D}}
\newcommand{\PtDp}{P_{\rm 3D^\prime}}
\newcommand{\vPsi}{\mathbf{\Psi}}
\newcommand{\vUpsilon}{\mathbf{\Upsilon}}
\newcommand{\vV}{\mathbf{V}}
\newcommand{\vW}{\mathbf{W}}
\newcommand{\vrhat}{\mathbf{\hat{r}}}
\newcommand{\vxhat}{\mathbf{\hat{x}}}
\newcommand{\vzhat}{\mathbf{\hat{z}}}
\newcommand{\vz}{\mathbf{z}}
\newcommand{\vpsi}{\mathbf{\psi}}
\newcommand{\vupsilon}{\mathbf{\upsilon}}
\newcommand{\bn}{\bar{n}}
\newcommand{\rhobar}{\bar{\rho}}
\newcommand{\vI}{\mathbf{I}}
\newcommand{\vH}{\mathbf{H}}
\newcommand{\vl}{\mathbf{l}}
\newcommand{\vL}{\mathbf{L}}
\newcommand{\vG}{\mathbf{G}}
\newcommand{\vD}{\mathbf{D}}
\newcommand{\vE}{\mathbf{E}}

%LyaF
%removed \ after several here - makes space before period - 
%add by hand when needed
\newcommand{\lya}{Ly$\alpha$}
\newcommand{\lyb}{Ly$\beta$}
\newcommand{\lyaf}{Ly$\alpha$ forest}
\newcommand{\vdf}{{\mathbf \delta_f}}
\newcommand{\vdF}{{\mathbf \delta_F}}
\newcommand{\lr}{\lambda_{{\rm rest}}}
\newcommand{\PF}{$P_F^{\rm 1D}(k_\parallel,z)$}
\newcommand{\bF}{\bar{F}}
\newcommand{\bC}{\bar{C}}
\newcommand{\bT}{\bar{T}}
\newcommand{\bS}{\bar{S}}

%journals
\newcommand{\mnras}{{\em Mon. Not. Roy. Astron. Soc. }}
\newcommand{\apjl}{{\em Astrophys. J. Let. }}
\newcommand{\apjs}{ApJS}
\newcommand{\jcap}{JCAP}
\newcommand{\physrep}{{\em Phys. Rept. }}
\newcommand{\aap}{{\em Astron. Astrophys. }}
\newcommand{\aj}{AJ }
\newcommand{\pasp}{PASP}
\newcommand{\pasj}{PASJ}
\newcommand{\uros}{Uro\v{s}}
\newcommand{\anze}{An\v{z}e}

\def\prd{{\em Phys. Rev. }{\bf D }}
\def\pr{{Phys.\ Rev.\ }}
\def\astropart{{Astro-particle Phys.~}}
\def\rvmp{{Rev.\ Mod.\ Phys.\ }}

\def\pvm#1{[PM: {\it #1}] }
\def\pvmhid#1{}
\def\af#1{[AF: {\it #1}] }
\def\afhid#1{}
\def\as#1{[AS: {\it #1}] }
\def\ashid#1{}
\def\hs#1{[HS: {\it #1}] }
%vb conflicts with \vb = \mathbf{b} used elsewhere
\def\vbh#1{[VB: {\it#1}] }
\def\cs#1{[CS: {\it#1}] }
\def\sb#1{[SB: {\it #1}] }
\def\zs#1{[ZS: {\it #1}] }
\def\zv#1{[ZV: {\it #1}] }
\def\uscomment#1{[US: {\it #1}] }
\def\referee#1{[REFEREE: {\it#1}] }


%comments by co-authors (I like better these than those in mydefinitions)
\newcommand{\AFR}[1]{{\color{red}AFR: #1}}


\begin{document}

\title{Parameterization of the \lya\ forest power spectrum}

\author{Andreu Font-Ribera \footnote{author list alphabetized}}
\email{a.font@ucl.ac.uk} 
\affiliation{Department of Physics and Astronomy, University College London, 
Gower Street, London, United Kingdom}
\author{Patrick McDonald}
\email{PVMcDonald@lbl.gov} 
\affiliation{Lawrence Berkeley National Laboratory, One Cyclotron Road,
Berkeley, CA 94720, USA}
\author{An\v{z}e Slosar}
\email{anze@bnl.gov} 
\affiliation{Brookhaven National Laboratory, Upton, NY 11973, USA}

\date{\today}

\begin{abstract}
Discuss unique role of \lya\ forest power spectrum, running and neutrino mass.
Discuss the need of emulators and expensive hydrodynamical simulations, and
the critical choice of likelihood parameterization.
Mention BOSS, eBOSS, DESI and high-resolution spectra.
\end{abstract}

\maketitle

\section{Introduction} 

Discuss here the unique window opened by Lyman-$\alpha$ (\lya) forest 
clustering, to study the linear power spectrum on small scales and redshifts 
higher than those available from galaxy surveys.
Discuss the role of hydrodynamic simulations in these studies, and the need 
for an emulator.
Discuss the importance of chosing the right parameterization in the emulator, 
since some cosmological parameters are not well measured by the \lya\ forest.
In particular, mention neutrino-mass degeneracy and the reasons to not use 
$\sigma_8$ defined at redshift zero.

Mention that even though this was studied already in \cite{McDonald2005a}, 
recent papers have ignored this issue \cite{Palanque-Delabrouille2015,
Yeche2017}. 
This will be relevant for future analyses, specially from DESI.

Even though past analyses have focused on the 1D power spectrum, 
the 3D power spectrum can also be measured \cite{Font-Ribera2018}, 
and it contains most of the information available from future surveys 
like DESI \cite{Font-Ribera2014}. 

In this paper we take another look at this topic, using modern hydrodynamic 
simulations and explicitely showing the accuracy of some of the approximations.
We start in section \ref{sec:over} with an overview of the different steps 
involved in a cosmological inference from the \lya\ power spectrum, and we 
continue in \ref{sec:like} with a description of the likelihood code, 
its user interface, and its internal parameterization. 
We describe the link between the likelihood and the \textit{emulator} in
section \ref{sec:emu}.
In section \ref{sec:sims} we discuss the simulations used in the emulator, 
and the post-processing of the snapshots. 
 


\section{Cosmological analyses with the \lya\ forest power spectrum}
\label{sec:over}

In this section we present an overview of the different aspects involved in 
a cosmological analysis of the small scale clustering from the \lya\ forest
power spectrum.

\begin{itemize}
 \item Measurement of the flux power spectrum: calibrate the quasar spectra, 
  fit the quasar continua, measure 2-point functions, covariance matrices 
  and possible contaminants.
 \item Infer the linear power spectrum: using hydrodynamical simulations, 
  build an emulator to translate the measured flux power to constraints 
  on the linear power spectrum of density at the redshift of the measurement
  (z ~ 3).
 \item Cosmological constraints: combine the inferred linear power spectrum
  with external datasets (primarily CMB studies) to constraint the parameters
  of a particular cosmological model.
\end{itemize}

Recently some studies have merged the last two steps, but we will argue that
this is not a good idea. 

Discuss here how does the measurement look like (redshift bins, k-range, 
typically in units of km/s...).

Discuss here the main difficulties in the emulator, including thermal history, 
mean flux, but also rescaling of optical depth. 

Discuss how the flux power mostly measures amplitude and slope around k=1, and 
it is only in combination with CMB that one can break degeneracies and 
constraint multiple cosmological parameters, including neutrino mass. 



\section{Degeneracies in linear theory}
\label{sec:lin}

Discuss the degeneracies in the context of linear theory, using model
predictions from CAMB.




\section{Simulations}
\label{sec:sims}

Describe here the simulations, include the initial conditions code (GenIC), 
the code to evolve the fields (MP-Gadget), the different boxes used, 
and the code to extract skewers (fake\_spectra).

Discuss also here the optical depth rescaling, different transfer functions
(if any), and thermal history effects (that will be mostly ignored in this
paper).


\subsection{Rescaling of the optical depth}

From each simulation we will get a set of snapshots, outputs at different 
redshifts. 

From each snapshot, we will extract \lya\ skewers, and use these to compute 
their power spectra. 
We will repeat this exercise for different rescalings of the optical depth, 
i.e., we will multiply the optical depth in all cells by a constant factor
in the range $0.8 < A_\tau < 1.2$ (approximately), and for each value of 
$A_\tau$ we will compute the transmitted flux fraction $F$, its mean value
(mean flux), and the power spectra of their fluctuations $\delta_F$. 

Therefore, from each snapshot we will get a set of power spectra for different
values of $A_\tau$.
We can label the different power spectra by their associated value of $A_\tau$,
or we can label them by their resulting value of the mean flux $\bar F$. 

\cite{Lukic2015} showed that the rescaling might introduce biases in the 
1D power spectrum for values of $A_\tau$ very different than one. 
However, their test compared two simulations with different thermal history
and different pressure, so it is difficult to tell whether the bias came 
from the rescaling or from the different IGM physics.

\AFR{It would be great to repeat this exercise in two simulations that 
have very similar thermal history but different mean flux, I will ask 
Jose Onorbe for help (he is visiting UCL soon). 
It is also possible that the test is clearer if we look at the 3D power,
where pressure only affects the high-k limit, and the scale independent
linera bias.}


\subsection{Rescaling of the temperature}

The Temperature-Density Relation (TDR) in the \lya\ forest can be reasonably
well described by a power law, 
\begin{equation}
 T(\rho) = T_0 \left(\frac{\rho}{\rho_0}\right)^{\gamma-1} ~,
\end{equation}
with a typical values for $\gamma$ between 1 and 1.6. 
If we use $\rho_0 = \bar \rho$, $T_0$ varies between 10,000 and 20,000K
\cite{Lukic2015}.

We can change the thermal history of the simulations by running the same box
with different \textit{TREECOOL} files, that contain the redshift evolution
of different heating and ionizing rates.

For each snapshot, we can fit their values of $T_0$ and $\gamma$, and use 
these to label the snapshot, instead of using the name of the TREECOOL, 
or the parameters that we have used to modify a given file 
(MP-Gadget can implement the recipes in \cite{Bolton2008} to modify TRECOOL
files by setting the parameters \textit{HeatAmplitude} and \textit{HealSlope}).

Just like we did with the optical depth, we can rescale the temperatures in 
post-processing. 
We could do this at two different levels:
\begin{itemize}
 \item Thermal broadening: when extracting the skewers from the boxes, 
  the last step is to compute the redshift-space-distorted optical depth. 
  To do that, the temperature at each cell is used to decide the local 
  smoothing that we need to apply, what is known as the thermal broadening. 
  It is trivial to take the same optical depth skewer, and convolve it for
  different temperature rescalings.
 \item Recombination rates: the temperature also sets the recombination rate,
  $\alpha(T) \sim T^{-0.7}$. 
  It would also be possible to change the recombination rate at each cell, 
  propagate this into a change in the neutral fraction (proportional to the 
  recombination rate), and finally to the optical depth (proportional to the 
  neutral fraction). 
\end{itemize}

Note that, even though this would capture most of the effects of having a 
different temperature, we would miss the effect that different temperature
have in the pressure smoothing.
However, we will include a parameter to study the dependence on the pressure
smoothing, the filtering length $k_F$, and this should be able to capture
the differences.


\subsection{Adding pressure smoothing}

The small scale structure is suppressed on very small scales because of the
pressure in the gas. 
As described in \cite{Hui1997,Gnedin1998}, the smoothing can be described
by a characteristic scale, $k_F)(z)$, the \textit{filtering scale}, that is
an integral version of the Jeans length that depends also on the temperature
in the past.

It would be interesting to see if we can add smoothing in postprocessing, 
effectively lowering the value of $k_F$ in the simulation by hand. 
Of course, it would not be possible to reduce the smoothing, and to do that 
one would need to run a simulation with an earlier redshift of reionization.

\AFR{We could also ask Jose Onorbe for help to setup this type of test. 
We would also need to discuss the best way to measure $k_F$ in the snapshots:
fit a Gaussian kernel in the power spectrum of $F_{\rm real}$, where no 
redshift-space distortions have been included? I believe that is what is 
used in all papers by Hennawi / Lukic / Onorbe.}


\subsection{Labelling the snapshots}

In the linear regime, and for a single specie, the growth of structure is 
scale independent, and it can be described by the growth factor $D(z)$.
%\begin{equation}
% P_L(z,k) = \frac{D^2(z)}{D^2(z_0)} ~ P_L(z_0,k) ~.
%\end{equation}

If the \lya\ power spectra depended only on the linear power spectrum, this 
would suggest that there would be a complete degeneracy between changing 
the overall amplitude of the linear power spectrum and changing the redshift
at which we ouput the snapshot. 
Therefore, we could use the amplitude of the linear power at a given snapshot
to label it, and use the different snapshots of the same simulation to study 
models with different amplitudes of the linear power. 

In section \ref{sec:dens} we will check whether this is still true at the 
level of the non-linear power spectrum, and in section \ref{sec:lya} we 
will look at this degeneracy in the \lya\ power spectrum. 

\AFR{How do we measure the linear power in the snapshot? 
I could see three options (in order of my preference): 
predict it using the power measured in the initial conditions, and the relative
growth as computed from CAMB/CLASS;
measure the density power in the snapshot, and fit the growth factor from 
the low-k part;
run a very cheap simulation without hydro and a very low value of $A_s$, to 
compute the actual linear power in the simulation (that might sadly differ 
from the predicted by CAMB/CLASS because of issues in the linear growht).}

To sum up, each snapshot will be used to generate multiple simulated 
fields, and we can label each of them by their values of: 
mean flux (1 parameter) $\bar F$,
linear power (3 parameters) $P_L$,
TDR (2 parameters) $T_0$ and $\gamma$, 
filtering scale $k_F$.

Redshift will NOT be a label describing the simulated field, and neither
will be the TREECOOL file or the redshift of reionization.
Since we will define the linear power in units of $\kms$, we will not 
need to include the Hubble parameter at the box as a label.
We will not care about any other cosmological parameter in the box
either.


\section{Degeneracies in the non-linear density power spectrum}
\label{sec:dens}

Use the simulations to measure the non-linear power spectrum of density
fluctuations, and show that the degeneracies are still there.


\section{Degeneracies in the flux power spectrum}
\label{sec:lya}

Show here the degeracies in the \lya\ 1D power (may be also 3D).



\section{Discussion}
\label{sec:disc}

Add here the discussion, and future work.


\section*{Acknowledgements}
AFR acknowledges support by an STFC Ernest Rutherford Fellowship, grant reference ST/N003853/1.
AS acknowledges hospitality of the University College London.
This work was partially enabled by funding from the UCL Cosmoparticle
Initiative.

\bibliography{main}
\bibliographystyle{JHEP}

%\appendix


\end{document}
