%\documentclass[11pt]{article}
\documentclass[preprintnumbers,prd,superscriptaddress,notitlepage,nofootinbib] {revtex4-1}

\usepackage{geometry, amsmath, amsthm, latexsym, amssymb, graphicx,wasysym}
\usepackage[dvipsnames]{xcolor}
\geometry{margin=1in, headsep=0.25in}
%\usepackage[usenames, dvipsnames]{color}

% define units, short-names...

\newcommand{\frachalf}{\frac{1}{2}}
\newcommand{\dzbar}{\delta_{\bar{z}}}
\newcommand{\fNL}{f_{\rm NL}}
\newcommand{\kNL}{k_{\rm NL}}
\newcommand{\FoM}{{\rm FoM}}
\newcommand{\cm}{\ \text{cm}}
\newcommand{\pc}{\ \text{pc}}
\newcommand{\Mpc}{\ \text{Mpc}}
\newcommand{\Gpc}{\ \text{Gpc}}
\newcommand{\kpc}{\ \text{kpc}}
\newcommand{\Gyr}{\ \text{Gyr}}
\newcommand{\hkpc}{\ h^{-1}\text{kpc}}
\newcommand{\hMpc}{\ h^{-1}\text{Mpc}}
\newcommand{\hMpcc}{\ h^{-3}\text{Mpc}^3}
\newcommand{\hGpc}{\ h^{-1}\text{Gpc}}
\newcommand{\hGpcc}{\ h^{-3}\text{Gpc}^3}
\newcommand{\ihMpc}{\ h\text{Mpc}^{-1}}
\newcommand{\iMpc}{\ \text{Mpc}^{-1}}
\newcommand{\Ms}{\ M_\odot}
\newcommand{\hMs}{\ h^{-1} M_\odot}
\newcommand{\eh}[1]{\exp{\left[#1\right]}}
\newcommand{\tim}[1]{\times 10^{#1}}%times 10 
\newcommand{\unit}[1]{\ \text{#1}}
\newcommand{\tb}[1]{\textcolor{blue}{#1}}
\newcommand{\tr}[1]{\textcolor{red}{#1}}
\newcommand{\tsc}[1]{#1}
\newcommand{\derd}{\,\mathrm{d}} % upright d for derivatives and integrals
\newcommand{\ddir}{\delta^\text{(D)}}
\newcommand{\dkron}{\delta^\text{(K)}}
\newcommand{\be}{\begin{equation}}
\newcommand{\ee}{\end{equation}}
\newcommand{\la}{\left\langle}
\newcommand{\ra}{\right\rangle}
\newcommand{\derivd}{\text{d}}
\renewcommand{\vec}{\bm}
\newcommand{\FT}[1]{{\rm FT}\left[#1\right]}
\newcommand{\twopicub}{\left(2\pi\right)^3}
\newcommand{\intvkop}{\int\frac{\derivd^3k}{\twopicub}}
\newcommand{\intvx}{\int \derivd^3x~}
\newcommand{\dqc}{\frac{\derivd^3q}{(2\pi)^3}}
\newcommand{\intvqop}{\int \dqc}
\newcommand{\dkpc}{\frac{\derivd^3k'}{(2\pi)^3}}
\newcommand{\dqcp}{\frac{\derivd^3q'}{(2\pi)^3}}
\newcommand{\dqcpp}{\frac{\derivd^3q''}{(2\pi)^3}}
\newcommand{\cyc}{\, \text{cyc.}}
\newcommand{\Dz}{\Delta z}
\newcommand{\Dv}{\Delta v}
\newcommand{\Dx}{\Delta x}
\newcommand{\Deta}{\Delta \eta}
\newcommand{\Dtheta}{\Delta \theta}
\newcommand{\bz}{\bar{z}}
\newcommand{\kms}{{\rm km~s^{-1}}}
\newcommand{\ikms}{{\rm s~km^{-1}}}
\newcommand{\kmsMpc}{\nicefrac[\rm]{km}{s\,Mpc}}
\newcommand{\summnu}{\Sigma m_\nu}
\newcommand{\Nnueff}{N_{\nu,{\rm eff}}}
\newcommand{\kmaxeff}{k_{\rm max,eff}}
\newcommand{\kmax}{{k_{\rm max}}}
\newcommand{\kmin}{{k_{\rm min}}}
\newcommand{\kNyq}{{k_{\rm Nyq}}}
\newcommand{\neff}{{n_{\rm eff}}}
\newcommand{\sL}{\mathcal{L}}
\newcommand{\sN}{\mathcal{N}}
\newcommand{\sH}{\mathcal{H}}
\newcommand{\sD}{\mathcal{D}}
\newcommand{\gmunu}{{g_{\mu \nu}}}
\newcommand{\keV}{{\rm keV}}
\newcommand{\GeV}{{\rm GeV}}
\newcommand{\erg}{\, {\rm erg}}
\newcommand{\gram}{\, {\rm g}}
\newcommand{\kelvin}{\, {\rm K}}
\newcommand{\deltahat}{\hat{\delta}}

%bias parameters
\newcommand{\bdelta}{b_\delta}
\newcommand{\bdtwo}{b_{\delta^2}}
\newcommand{\bstwo}{b_{s^2}}
%renormalized fields
\newcommand{\msdtwo}{\left[\delta^2\right]}
\newcommand{\msstwo}{\left[s^2\right]}
%bold faced vectors
\newcommand{\vtheta}{\mathbf{\theta}}
\newcommand{\vDtheta}{\mathbf{\Delta \theta}}
\newcommand{\vv}{\mathbf{v}}
\newcommand{\vnabla}{\mathbf{\nabla}}
\newcommand{\veta}{\mathbf{\eta}}
\newcommand{\vk}{\mathbf{k}}
\newcommand{\vq}{\mathbf{q}}
\newcommand{\vt}{\mathbf{t}}
\newcommand{\vb}{\mathbf{b}}
\newcommand{\vx}{\mathbf{x}}
\newcommand{\vc}{\mathbf{c}}
\newcommand{\vone}{\mathbf{1}}
\newcommand{\vmu}{\mathbf{\mu}}
\newcommand{\vzeta}{\mathbf{\zeta}}
\newcommand{\vn}{\mathbf{n}}
\newcommand{\vgamma}{\mathbf{\gamma}}
\newcommand{\vpi}{\mathbf{\pi}}
\newcommand{\vrho}{\mathbf{\rho}}
\newcommand{\vphi}{\mathbf{\phi}}
\newcommand{\vchi}{\mathbf{\chi}}
\newcommand{\vomega}{\mathbf{\omega}}
\newcommand{\vu}{\mathbf{u}}
\newcommand{\vj}{\mathbf{j}}
\newcommand{\vg}{\mathbf{g}}
\newcommand{\vm}{\mathbf{m}}
\newcommand{\vd}{\mathbf{d}}
\newcommand{\vdelta}{\mathbf{\delta}}
\newcommand{\vy}{\mathbf{y}}
\newcommand{\vf}{\mathbf{f}}
\newcommand{\vp}{\mathbf{p}}
\newcommand{\vep}{\mathbf{\epsilon}}
\newcommand{\vo}{\mathbf{o}}
\newcommand{\vs}{\mathbf{s}}
\newcommand{\vsh}{\hat{\mathbf{s}}}
\newcommand{\vS}{\mathbf{S}}
\newcommand{\vC}{\mathbf{C}}
\newcommand{\vXi}{\mathbf{\Xi}}
\newcommand{\vQ}{\mathbf{Q}}
\newcommand{\vDelta}{\mathbf{\Delta}}
\newcommand{\vP}{\mathbf{P}}
\newcommand{\vN}{\mathbf{N}}
\newcommand{\vA}{\mathbf{A}}
\newcommand{\vM}{\mathbf{M}}
\newcommand{\vK}{\mathbf{K}}
\newcommand{\vB}{\mathbf{B}}
\newcommand{\vU}{\mathbf{U}}
\newcommand{\vT}{\mathbf{T}}
\newcommand{\vF}{\mathbf{F}}
\newcommand{\br}{\mathbf{r}}
\newcommand{\vR}{\mathbf{R}}
\newcommand{\Tr}{\mathrm{Tr}}
\newcommand{\lnL}{{\mathcal{L}}}
\newcommand{\vxperp}{\mathbf{x_\perp}}
\newcommand{\vkperp}{\mathbf{k_\perp}}
\newcommand{\kperp}{k_\perp}
\newcommand{\vrperp}{\mathbf{r_\perp}}
\newcommand{\vpar}{v_\parallel}
\newcommand{\rpar}{r_\parallel}
\newcommand{\kpar}{k_\parallel}
\newcommand{\tk}{\tilde{k}}
\newcommand{\tmu}{\tilde{\mu}}
\newcommand{\PtD}{P_{\rm 3D}}
\newcommand{\PtDp}{P_{\rm 3D^\prime}}
\newcommand{\vPsi}{\mathbf{\Psi}}
\newcommand{\vUpsilon}{\mathbf{\Upsilon}}
\newcommand{\vV}{\mathbf{V}}
\newcommand{\vW}{\mathbf{W}}
\newcommand{\vrhat}{\mathbf{\hat{r}}}
\newcommand{\vxhat}{\mathbf{\hat{x}}}
\newcommand{\vzhat}{\mathbf{\hat{z}}}
\newcommand{\vz}{\mathbf{z}}
\newcommand{\vpsi}{\mathbf{\psi}}
\newcommand{\vupsilon}{\mathbf{\upsilon}}
\newcommand{\bn}{\bar{n}}
\newcommand{\rhobar}{\bar{\rho}}
\newcommand{\vI}{\mathbf{I}}
\newcommand{\vH}{\mathbf{H}}
\newcommand{\vl}{\mathbf{l}}
\newcommand{\vL}{\mathbf{L}}
\newcommand{\vG}{\mathbf{G}}
\newcommand{\vD}{\mathbf{D}}
\newcommand{\vE}{\mathbf{E}}

%LyaF
%removed \ after several here - makes space before period - 
%add by hand when needed
\newcommand{\lya}{Ly$\alpha$}
\newcommand{\lyb}{Ly$\beta$}
\newcommand{\lyaf}{Ly$\alpha$ forest}
\newcommand{\vdf}{{\mathbf \delta_f}}
\newcommand{\vdF}{{\mathbf \delta_F}}
\newcommand{\lr}{\lambda_{{\rm rest}}}
\newcommand{\PF}{$P_F^{\rm 1D}(k_\parallel,z)$}
\newcommand{\bF}{\bar{F}}
\newcommand{\bC}{\bar{C}}
\newcommand{\bT}{\bar{T}}
\newcommand{\bS}{\bar{S}}

%journals
\newcommand{\mnras}{{\em Mon. Not. Roy. Astron. Soc. }}
\newcommand{\apjl}{{\em Astrophys. J. Let. }}
\newcommand{\apjs}{ApJS}
\newcommand{\jcap}{JCAP}
\newcommand{\physrep}{{\em Phys. Rept. }}
\newcommand{\aap}{{\em Astron. Astrophys. }}
\newcommand{\aj}{AJ }
\newcommand{\pasp}{PASP}
\newcommand{\pasj}{PASJ}
\newcommand{\uros}{Uro\v{s}}
\newcommand{\anze}{An\v{z}e}

\def\prd{{\em Phys. Rev. }{\bf D }}
\def\pr{{Phys.\ Rev.\ }}
\def\astropart{{Astro-particle Phys.~}}
\def\rvmp{{Rev.\ Mod.\ Phys.\ }}

\def\pvm#1{[PM: {\it #1}] }
\def\pvmhid#1{}
\def\af#1{[AF: {\it #1}] }
\def\afhid#1{}
\def\as#1{[AS: {\it #1}] }
\def\ashid#1{}
\def\hs#1{[HS: {\it #1}] }
%vb conflicts with \vb = \mathbf{b} used elsewhere
\def\vbh#1{[VB: {\it#1}] }
\def\cs#1{[CS: {\it#1}] }
\def\sb#1{[SB: {\it #1}] }
\def\zs#1{[ZS: {\it #1}] }
\def\zv#1{[ZV: {\it #1}] }
\def\uscomment#1{[US: {\it #1}] }
\def\referee#1{[REFEREE: {\it#1}] }


%comments by co-authors (I like better these than those in mydefinitions)
%(italics was probably inheritted from Jordi from black and white printer 
%days...)
\newcommand{\AFR}[1]{{\color{red}AFR: #1}}
\renewcommand{\as}[1]{{\color{blue}AS: #1}}
%I wonder what I did to deserve ``olive"... I guess it is ok now that I'm 
%getting used to it... 
%What do you mean? I love olive! I first tried "green", but it was too bright to read. 
%They didn't have darkgreen, so I went for olive :-)
\renewcommand{\pvm}[1]{{\color{olive}PVM: #1}}
\def\todo#1{[TO DO: {\it #1}] }
\newcommand{\NEW}[1]{{\color{violet}AFR: #1}}


\begin{document}

\title{Parameterization of the \lya\ forest power spectrum}

\author{Andreu Font-Ribera} % \footnote{author list alphabetized}}
\email{a.font@ucl.ac.uk} 
\affiliation{Department of Physics and Astronomy, University College London, 
Gower Street, London, United Kingdom}
%\author{Patrick McDonald}
%\email{PVMcDonald@lbl.gov} 
%\affiliation{Lawrence Berkeley National Laboratory, One Cyclotron Road,
%Berkeley, CA 94720, USA}
%\author{An\v{z}e Slosar}
%\email{anze@bnl.gov} 
%\affiliation{Brookhaven National Laboratory, Upton, NY 11973, USA}

\date{\today}

\begin{abstract}
This is an internal document aimed at clarifying the work-flow related to extracting cosmological information 
from the \lya\ forest power spectrum, using emulators and hydrodynamical simulations.
It could be used for an eventual publication, but for now the main goal is to make sure we are all on the
same page, and help build the science case for the 11th DiRAC Call.
Some of this discussion started years ago when I was trying to understand the details of the SDSS-I analyses
by Pat and \uros, and how one would do an equivalent 3D analysis. 
Some of it is from chats with Pat and Simeon in 2016, trying to understand why $H_0$ should not be a
parameter in \lya\ analyses. 
The current version is mostly influenced by discussions with \anze\ in a whiteboard of the Cosmoparticle hub,
and by feedback from Pat on an earlier version of this document.
\end{abstract}

\maketitle

\section{Introduction} 

Discuss here the unique window opened by Lyman-$\alpha$ (\lya) forest 
clustering, to study the linear power spectrum on small scales and redshifts 
higher than those available from galaxy surveys.
Discuss the role of hydrodynamic simulations in these studies, and the need 
for an emulator.
Discuss the importance of chosing the right parameterization in the emulator, 
since some cosmological parameters are not well measured by the \lya\ forest.
In particular, mention neutrino-mass degeneracy and the reasons to not use 
$\sigma_8$ defined at redshift zero.

Mention that even though this was studied already in \cite{McDonald2005a}, 
recent papers have ignored this issue \cite{Palanque-Delabrouille2015,
Yeche2017}. 
This will be relevant for future analyses, specially from DESI.

Even though past analyses have focused on the 1D power spectrum, 
the 3D power spectrum can also be measured \cite{Font-Ribera2018}, 
and it contains most of the information available from future surveys 
like DESI \cite{Font-Ribera2014}. 

In this paper we take another look at this topic, using modern hydrodynamic 
simulations and explicitely showing the accuracy of some of the approximations.
We start in section \ref{sec:over} with an overview of the different steps 
involved in a cosmological inference from the \lya\ power spectrum, and we 
continue in \ref{sec:like} with a description of the likelihood code, 
its user interface, and its internal parameterization. 
We describe the link between the likelihood and the \textit{emulator} in
section \ref{sec:emu}.
In section \ref{sec:sims} we discuss the simulations used in the emulator, 
and the post-processing of the snapshots. 
 


\section{Cosmological analyses with the \lya\ forest power spectrum}
\label{sec:over}

In this section we present an overview of the different aspects involved in 
a cosmological analysis of the small scale clustering from the \lya\ forest
power spectrum.

\begin{itemize}
 \item Measurement of the flux power spectrum: calibrate the quasar spectra, 
  fit the quasar continua, measure 2-point functions, covariance matrices 
  and possible contaminants.
 \item Infer the linear power spectrum: using hydrodynamical simulations, 
  build an emulator to translate the measured flux power to constraints 
  on the linear power spectrum of density at the redshift of the measurement
  (z ~ 3).
 \item Cosmological constraints: combine the inferred linear power spectrum
  with external datasets (primarily CMB studies) to constraint the parameters
  of a particular cosmological model.
\end{itemize}

Recently some studies have merged the last two steps, but we will argue that
this is not a good idea. 

Discuss here how does the measurement look like (redshift bins, k-range, 
typically in units of km/s...).

Discuss here the main difficulties in the emulator, including thermal history, 
mean flux, but also rescaling of optical depth. 

Discuss how the flux power mostly measures amplitude and slope around k=1, and 
it is only in combination with CMB that one can break degeneracies and 
constraint multiple cosmological parameters, including neutrino mass. 



\section{Public likelihood code} \label{sec:like}

In order to properly discuss our choice of parameterization and simulation
setup, it will be helpful to start by describing how we intend the end
product, the \textit{likelihood code}, to be used in a cosmological analysis.


\subsection{User interface}

\ashid{I don't like the expression ``user interface'' here. Wouldn't
  likelihood parameterization be more appropriate?}
\afhid{\anze, this is an internal document, to make sure we are all (including
Chris, Hiranya, Andrew, Tom...) in the same page.
The second part of this section is about the parameterization, this first
part is really just about the interface...
As Pat mentiones below, we can give an easy interface to the users,
and under the hood we compress that into any set of parameters we want.}

The aim of the project is to provide a public likelihood package that others
can use to include \lya\ results in their cosmological parameter constraints.
The end-user does not need to know about the internal details, about the
suite of simulations, the nuisance parameters or the interpolation scheme
used in the emulator.
They do not need to know either whether internally we are describing the
power in units of $\kms$, $\Mpc$ or $\hMpc$, etc.
There are different possible interfaces that we could setup, and probably
we will want to provide more than one with different levels of complexity.
But we will start by discussing a particular interface, where we will ask
the user to provide for each cosmological model:
\begin{itemize}
 \item $P(z,k)$, the linear density power spectrum
  \footnote{During the rest of this paper we will use linear power spectrum
  to refer to the baryons+CDM power.}, as a function of redshift and
  wavenumber, in units of $\iMpc$.
  The redshift range should cover at least $2 < z < 5$, and the wavenumber
  range should cover at least $0.01 \iMpc < k < 10 \iMpc$.
 \item $P_\theta(z,k)$, the linear power spectrum of velocity divergence,
  or \textit{velocity power}, over the same range of redshift and scales.
 \item Hubble parameter as a function of redshift, $H(z)$, over the same
  redshift range.
 \item If we wanted the likelihood to be able to describe 3D clustering as well,
  we would need as an input the angular diameter distance $D_A(z)$ (or the
  Alcock-Paczy\'nski coefficient, $\propto D_A(z) H(z)$, \cite{Alcock1979})
  over the full redshift range, as well as the sound horizon at the drag epoch
  describing BAO, $r_d$. \AFR{Although we could also fit $r_d$ ourselves 
  from the linear power ourselves under the hood, to prevent having some 
  users using EH98 and others using CAMB.}
\end{itemize}

\pvmhid{I don't think you should skimp on what you want from the outside. E.g., 
in neutrino mass era it is starting to feel a little quaint to talk 
about $D(z)$, 
$f(z)$ -- both generally depend on $k$. At the same time, producing density 
and velocity power spectra as a function of k and z is just getting easier and 
easier as people get more comfortable with things like CLASS. 
(I mean, no I 
don't actually think $k$ dependence of $D$ and $f$ will ever matter, 
but it seems
just easier to ask for the linear $P(z,k)$ than worry about defining and
justifying these things
(e.g., I remember kind of
laughing at Alexey Makarov for producing $D(z)$ by running CMBfast, which 
seemed like overkill when I had a little numerical integrator code, but to him
this was actually easier than setting up that code...))
I wouldn't worry
too much about what you're going to ask for though. Focus on making a good 
emulator and then ask for what you need for that... 
}
\afhid{Great, that's all I wanted to hear. For now, let's assume this is a 
possible option, and we'll come back to it later on.} 
\ashid{I like Pat's idea of just asking for a bunch of power spectra,
  including matter, velocity and perhaps baryon as well. For example,
  you could easily also check if the power spectra are not so
  pathological that approximations employed are not valid. For example
  Pat's code wasn't the right thing to use for WDM models and here you
  can guard against that. }

The user will also specify:
\begin{itemize}
 \item Data products to use:
  SDSS-I from \cite{McDonald2006},
  BOSS from \cite{Palanque-Delabrouille2013},
  HIRES/UVES from \cite{Viel2013},
  XQSO-100 from \cite{Irsic2017},
  HIRES/UVES from \cite{Walther2018a}.
  We should also allow the user to specify what redshifts to use from each
  dataset, since they are independent, and probably also allow the user
  to specify the scales to be used in the fit for each dataset.
 \item Extra analysis settings, like whether to allow for running in the
  linear power, differences in the linear growth, or differences in the BAO
  wiggles.
\end{itemize}

\AFR{It is not clear to me whether at this point the user would also be able
to set other settings of the likelihood or not, like the way we treat
contamination by DLAs or metals, resolution or noise corrections, or the
parameterization of the temperature-density relation in the simulations.}
\as{This is implementational detail, no need to worry about this now.}

\AFR{I believe the answer is that each of these analysis choices would be
a different likelihood object, and then the user can decided whether to
use the likelihood object where the DLA marginalization used the formula
from \cite{McDonald2005b} or whether to use the one that used the formula
from \cite{Rogers2018a}.
Similarly, every time we want to use a different prior for the nuisance
parameters (say we want to include measurements of mean flux or temperature)
we would need to recompute the marginalization, and provide a new likelihood
look-up table.}

The output from the \textit{likelihood code} will be:
\begin{itemize}
 \item A value of (log-)likelihood for each dataset, possibly a value for 
  each redshift bin. 
  Most users will only care about this.
 \item For the experts, we would also output the best-fit theoretical 
  prediction for each dataset, for that particular cosmological model.
  We would also provide the values of the nuisance parameters that correspond
  to the best fit model (mean flux, temperature-density relation, metal or DLA
  contamination...).
 \item We could also provide a random sample of theory lines that are above a 
  likelihood threshold for that particular model, exploring different 
  thermal histories and other nuisance parameters.
\end{itemize}

\AFR{Actually, I don't think at this point you can get any of the above.
The user of the final likelihood code will only get a total value for the
likelihood for the total dataset.
Even though one would compute the un-marginalized likelihood for each redshift
and each dataset separately, at the step of marginalizing over nuisance 
parameters (see section \ref{sec:emu}) you would need to include them 
all at the same time.}
\as{No, no. You don't marginalise over nuissance parameters inside
  your likelihood code. Some nuissance parameters will have particular
  degeneracies with the big picture cosmological parameters, and in
  addition you never know how clever the user's sampler is. It is ok
  to ask for linear power, $f$, $H(z)$, $D_A(z)$ and a set of
  nuissance parameters, which users doesn't need to understand except
  for valid ranges. You can add a small wrapper that does this
  marginalisation internally for those who don't care.}
\pvm{In principle I more or less agree with \anze\ about the nuisance 
parameters, the core level of the code should be written treating all 
parameters symmetrically and this should be kept accessible... although,
if you were, e.g., importance sampling a Planck chain you would effectively
return to needing the code to marginalize over nuisance parameters. On the 
other hand, I remember long ago in Alexey Makarov days we ended up retreating
from this idea because the generic MCMC was painfully slowed by marginalization
over these nuisances when we knew how to make it fast internally... Ugh - 
I'm going to try not to read anything but emulator section since I think this
discussion of final public likelihood code is getting way ahead of yourself... 
\smiley\
I'd focus on ``what do I need to do to get an emulator capable of any kind of
fit to a LyaF data set?" ...  }

\AFR{Nooooooo! I need to understand how the final package will be used, 
before I can think of emulators and simulations!}

\AFR{Pat, in your 2005 paper, you had $\chi^2$ look-up tables, that were only
a function of amplitude and slope (and may be running). 
I always thougth that this type of look-up table would be the end product of 
the analysis, with a nice wrapper around it with a public interface to 
translate cosmology to our cryptic parameters in the lookup table.
Isn't it the case? 
You made the look-up table public, and that table had already marginalized
over all nuisance parameters, right?}
\pvm{Yes. This is getting at what I rambled about in comment later. While
\anze's desired option to control all parameters simultaneously in a cosmoMC
run should certainly *exist,* I don't think you should present this as, e.g., 
what anyone would run in their first use of your results. I do think you still
want to produce some form of ``de-forested'' linear power main result, that
will contain everything most people would want. You need it for 
``opening the black box" purposes, but also, it will be faster, and why force
people to mess with a bunch of forest nuisance parameters they don't care 
about when you know they can get the same results without it? The option to
run global chain with LyaF nuisances should exist for comparison, and so you
can see correlations, but not be the lead option... (this is sort of what I
was trying to indicate here saying ``core level of code", but I didn't really
think through how necessary I think the marginalized version really is...).
... I don't think this has to be very ``cryptic" even under the hood though...
I wouldn't call a $\chi^2$ table in $\Delta^2_L, \neff(z=3,k=0.009 s/km)$ 
cryptic... } 

\NEW{Perfect. Just for the record, in my original text (two days ago) I never
meant to talk about the CosmoMC option, I always had in mind to compress it
all to a look-up table with $\Delta^2_L, \neff(z=3,k=0.009 s/km)$ or 
equivalent (may be extending to $f_\star$ or running.}

\pvm{Ok, since it isn't actually perfectly trivial to define 
$\Delta^2_L, \neff(z=3,k=0.009 s/km)$ in practice, e.g., do you really want to
do an infinitesimal derivative at exactly this point or some kind of broader
thing, I think you will always want to control how that is done by asking 
for full P(k), and they should be happy to let you take care of it, so maybe it
could get a little cryptic, but this is not much related to the more physical
stuff related to emulator.} 
\NEW{Agreed.}
\as{I agree as well.}



\subsection{From cosmological model to likelihood parameters}

Under the hood, we will use more effective (and cryptic) parameters in our 
likelihood, to reduce the internal degeneracies between parameters.
There are many, many possibilities here, but we will start by discussing a 
possible setting. 

\subsubsection{Fiducial cosmological model}
 
We will choose a fiducial cosmological model, based on a recent Planck+BAO 
analysis, and use it to compute a fiducial linear power spectrum, $P_L^0(z,k)$, 
a fiducial Hubble expansion, $H^0(z)$, a fiducial growth rate $f^0(z)$...
All quantities with a superscript $^0$ will refer to the fiducial model.

\subsubsection{Linear power shape}

Since we have decided to use only models where the linear power spectrum 
can be factorized, we will describe its shape at the central redshift only, 
$z_\star=3$. 
In general we will use the subscript $_\star$ to refer to quantities that 
have been evaluated at $z_\star$, but in this sub-section we will ignore 
the redshift and assume that all power spectra are evaluated at $z_\star$.

We will compress the difference between the input power spectrum, $P_L(k)$, 
and the fiducial power spectrum, $P^0_L(k)$, into a handful of parameters. 
We start by fitting the fiducial power with a smooth function, 
$P^0_{nw}(k)$, using the \textit{no-wiggle} model from \cite{Eisenstein1998}.
We define the oscillatory (or \textit{wiggle}) component of the 
fiducial power, $W^0(k)$, as the ratio of these two powers:
\begin{equation}
 W^0(k) = \frac{P^0_L(k)}{P^0_{nw}(k)} ~.
\end{equation} 
\pvm{Are your boxes actually going to be big enough to worry about BAO? If 
so, I think you should *definitely* model that part by linear bias/more 
general PT... you can use that to pull out the feature essentially 
analytically. Never mind what I said about ``units of BAO"... this is surely 
the way to go since you don't want to be trying to interpolate between raw
sim power at that scale anyway...}
\NEW{No, in the main analysis I would not vary $r_d$ as a free parameter. 
But talking about this option might help others understand what we are doing, 
even though we then say: "we do not expect our results to be sensitive to
BAO, and therefore we fix $r_d$ to Planck value".} 

We will assume that the oscillatory component of the input model can be 
described with the oscillations in the fiducial model, shifted by the ratio 
of their sound horizons at the drag epoch ($r_d$):
\begin{equation}
 W(k) \sim W^0(\beta k)  
\qquad  
 \beta = \frac{r_d}{r^0_d} ~.
\end{equation}
We have decided to use $\beta$ and not $\alpha$, more common in BAO analyses,
because the latter includes a ratio of transformations from observable to 
commoving coordinates, and we do not need this at this point. 

Finally, we will model the differences in the smooth component with a smooth,
parameterized function:
\begin{equation}
 B(k) = \frac{P_{nw}(k)}{P_{nw}^0(k)} ~.
\end{equation}
There are different parameterizations possible, but for the rest of this 
discussion we will assume that we use three parameters: an amplitude, 
a slope, and a running of the slope, evaluated around $k_p = 1 \iMpc$. 

To summarize, we will describe the input linear power at $z_\star$ as:
\begin{equation} \label{eq:Pk_param}
 P_L(k) = B(k) ~ P_{nw}^0(k) ~ W^0(\beta k) ~.
\end{equation}

\subsubsection{Linear growth}

We will compute the difference of the input logarithmic growth rate with that 
in the fiducial cosmology, $\Delta f_\star = f(z_\star) - f^0(z_\star)$, 
evaluated at $z_\star=3$. 
We will approximate that the different growth rate at $z_\star$ is enough 
to compute the difference in linear growth at any redshift (within the range):
\begin{equation}\label{eq:growth}
 \frac{D(z)}{D^0(z)} = 1 + \Delta f_\star ~ \frac{\Delta z}{1 + z_\star} ~.
\end{equation} 
Note that in LCDM models, and at $2 < z < 5$, the differences in growth rate 
are typically less than 1\%, as shown in Figure \ref{fig:fz_Om}.

\subsubsection{Hubble expansion}

If we could observe the \lya\ power spectrum in comoving coordinates, that 
would be enough. 
However, we observe the power spectrum in observing coordinates, wavelengths
and angles, and a more natural choice is to use velocity units ($\kms$) for 
the clustering measurements. 
Indeed, all recent measurements of the 1D \lya\ power reported their results
in units of $\kms$, and we will assume the same in this discussion.

In general, we would need to use $H(z)$ from each model to compare 
measurements in $\kms$ with model preditions in $\Mpc$:
\begin{equation}
 q = \frac{1+z}{H(z)} ~ k = a_v k~,
\end{equation}
where we use $q$ to refer to wavenumbers in velocity units, and we have 
defined $a_v$ as the transformation from $\kms$ to $\Mpc$.
This would force us to add in the emulator some sort of Hubble 
parameter, either at $z=0$ ($h$) or at $z_\star=3$. 
However, as suggested by \cite{McDonald2005a}, it is possible to avoid this
burden if we describe our model (the linear power spectrum) already in 
units of $\kms$. \as{This confused the shit out of me until very
  recently. I vote for having power spectra in Mpc and also require
  $H(z_\star)$ and do this conversion internally -- it is trivial. }
\AFR{Yes, \anze, that was the entire point of the user interface. 
We ask the users to provide power in $\Mpc$, and we ask for $H(z)$, 
and we do the convertions under the hood.}

We claim that two models with different expansion histories $H(z)$, but the
same linear power in units of $\kms$, will have very similar \lya\ power
spectra, with small remaining differences being caused by astrophysical 
effects (different reionization history, different thermal history, different
mean flux...). 
And since we plan to marginalize over these to get the final cosmological 
constraints, we do not need to worry about these differences. 
For the rest of this discussion, we will assume that this is true.

\pvm{I agree with \anze\ that you should move to just using Mpc and asking for
$H(z)$, where, a clincher is in 3D you can't get
away from asking for at least $(H D_A)(z)$. The argument about km/s was always
that you should quote results as a linear power measurement in km/s at 
$z\sim 3$, *if you are going to quote results from LyaF $\sim$alone* -- the
goal there was to produce ``model independent" constraints that could be
propagated forward.}
\NEW{My plan was to translate the flux power into linear power, in $\kms$, 
and make that public as a look-up table, with a user interface in $\Mpc$ to 
make it easier for others to use. 
This is what I was trying to explain all along, mostly in section
\ref{sec:over}, but I guess I didn't do a good job \smiley.} \pvm{But if 
you're going to be using flux power directly in global chains fitting 
cosmological parameters, which seems to be what you're talking about here, 
that argument doesn't apply -- you have the only relevant cosmological model 
on hand point-by-point. }
\NEW{I never talked about doing this! Only \anze\ talked about it. 
I only talked about cosmological models as a wrapper, but the likelihood 
and the emulator would only hear about our compressed parameterization.} 
\pvm{This does raise the question though, how much you 
want to commit to that approach, i.e., to primarily present a likelihood code
that does direct fits to flux power measurements on the fly, vs. boiling 
things down to linear power measurement (which you would again obviously 
present in observable coordinates). Certainly internally you want to be able
to do both, and the code to do linear boil-down should probably wrap the other
one, but there is a question where to spend more time cleaning and advertising. 
Arguments in favor of boil-down to linear could be speed (in old days, like 
I said, we could do both, but fit to flux power really slowed down global
fits, while my $\chi^2$ table gave identical results), and just... it is nice
to be able to show the more-or-less LyaF-model-independent cosmological thing
you claim to have measured, not just present a big black likelihood box. 
It helps, e.g., estimate how these constraints will affect new cosmological  
models. In any case, you can figure out what to advertise later. It seems like
first goal should be to produce $\chi^2$ contours for, e.g.,  
$\Delta^2,\neff(z=3, k=0.009 s/km)$,  
defined as deviations from a central model (i.e.,
effectively variations of $A_s$ and $n_s$ with other parameters fixed), 
to compare to past... this is clearly going to be *most of* what matters.} 
\NEW{Yes, I agree with the above.}


How does this affect the discussion above?

Let us use $\tilde P(q)$ to refer to power spectra in units of velocity:
\begin{equation}
 \tilde P(q) = a_v^3 ~ P(k= q / a_v) ~.
\end{equation}

We can then redo the whole discussion above, but using power spectra in 
velocity units:
\begin{align} 
 \tilde P^0_L(q) & = (a^0_v)^3 ~ P^0_L(q / a^0_v)         \nonumber \\
   & = (a^0_v)^3 ~ P^0_{nw}(q / a^0_v) ~ W^0(q / a^0_v)      \nonumber \\
   & = \tilde P^0_{nw}(q) ~ \tilde W^0(q)  ~,
\end{align}
where we have also defined $\tilde W^0(q) = W^0(q / a^0_v0$.

We can now define a term for the ratio of the smooth power, in velocity
units:
\begin{align}
 \tilde B(q) & = \frac{\tilde P_{nw}(q)}{\tilde P^0_{nw}(q)}    \nonumber \\
  & = \left(\frac{a_v}{a_v^0}\right)^3 
      \frac{P_{nw}(q / a_v)}{P^0_{nw}(q / a_v^0)} ~,
\end{align}
and we can finally write 
\begin{align}
 \tilde P_L(q) & = \tilde P_{nw}(q) ~ W(q / a_v)                \nonumber \\
  & = \tilde B(q) ~ \tilde P^0_{nw}(q) ~ \tilde W^0(\alpha q) ~,
\end{align}
where we have defined $\alpha$ as the ratio of sound horizons in units of 
velocity: 
\begin{equation}
 \alpha = \beta \frac{a_v^0}{a_v} = \frac{r_d ~ H_\star}{r_d^0 ~ H^0_\star}~.
\end{equation}

The cosmological model in the likelihood will then be described by a 
set of parameters $\theta$ describing the linear power spectrum, including:
\begin{itemize}
 \item Ratio of sound horizons in units of $\kms$, $\alpha$, where the 
conversion from $\Mpc$ to $\kms$ is computed at $z_\star=3$. 
This is the inverse of the usual definition of BAO $\alpha_\parallel$.
 \item Approximately 3 parameters describing the ratio of the smooth
  linear power at $z_\star=3$, in units of $\kms$.
 \item Difference of growth rates at $z_\star=3$, $\Delta f_\star$. 
\end{itemize}

We will ask the likelihood code: for these set of parameters $\theta$,
what is the likelihood of getting the measured power, after marginalizing
over all nuisance parameters (including mean flux and thermal history)?
Or in math, what we want is:
\begin{equation} \label{eq:marg}
 L(\vd | \theta)
  = \int d\phi ~ \Pi(\phi) ~ L(\vd | \theta, \phi) ~,
\end{equation}
where $\vd$ is the measured flux power spectrum, $\phi$ are the nuisance
parameters, and $\Pi(\phi)$ are the priors on the nuisance parameters.
\as{Again, marginalisation is for the user to do. The code should
  output total $\chi^2$ at this point and optionally data-points and
  theory predictions at this point include full gory of nuissance
  parameters. If you look at eg \texttt{cosmomc} it internally
  marginalises already over some 15 Planck parameters, it can do 15
  more for us.}
\AFR{Ok, this is the type of discussion I wanted to have. 
Is this really how it was done in SDSS-I? What was the look-up table for, then?
To make plots?}
\pvm{I guess I should further to un-agree with \anze... 
If you wanted to assume everyone
will be using cosmoMC so you could just make a module for that and broadcast
it, I guess it would be ok (but I think still a lot of people would consider
your LyaF parameters to be too literally a nuisance... and be less tolerant 
of it than of Planck nuisance), but I don't think you want to think this way.
I don't think one MCMC code will dominate, so you need to assume people are
going to be grafting your likelihood code into various things themselves, 
so you want to make sure to have a very easy option. }
\NEW{Agreed. We should have an easy version for all to use, and a difficult
one for testing and for the experts to use.}

This likelihood will have been evaluated at a grid of points in $\theta$,
and it will be stored as a look up table.
Evaluating the likelihood, once the lookup table has been computed,
should then be trivial and extremely fast.

  
\afhid{For simplicity, we will restrict ourselves to models with a linear power 
  spectra that can be factorized in a power spectrum at a central redshift, 
  $P(z_\star=3,k)$, and a scale independent growth factor, $D(z)$.}
\ashid{If you are to compress flux measurement over X redshift bins
  into one linear power constrain, just ask for the linear power
  at that $z_{\rm pivot}$ and $f$ at that redshift. If you are going
  to compress into two $z$s, ask for those two $z$s and two $f$s}.
\afhid{Yes, this is an option. But as Pat mentions below, this might not 
 be the easier option for some users. But when I mentioned above that we 
 could have more than one interface, this is what I had in mind.}

\pvmhid{I don't think you should skimp on what you want from the outside. E.g., 
in neutrino mass era it is starting to feel a little quaint to talk 
about $D(z)$, 
$f(z)$ -- both generally depend on $k$. At the same time, producing density 
and velocity power spectra as a function of k and z is just getting easier and 
easier as people get more comfortable with things like CLASS. 
(I mean, no I 
don't actually think $k$ dependence of $D$ and $f$ will ever matter, 
but it seems
just easier to ask for the linear $P(z,k)$ than worry about defining and
justifying these things
(e.g., I remember kind of
laughing at Alexey Makarov for producing $D(z)$ by running CMBfast, which 
seemed like overkill when I had a little numerical integrator code, but to him
this was actually easier than setting up that code...))
I wouldn't worry
too much about what you're going to ask for though. Focus on making a good 
emulator and then ask for what you need for that... 
}
\afhid{Great, that's all I wanted to hear. For now, let's assume this is a 
possible option, and we'll come back to it later on.} 
\ashid{I like Pat's idea of just asking for a bunch of power spectra,
  including matter, velocity and perhaps baryon as well. For example,
  you could easily also check if the power spectra are not so
  pathological that approximations employed are not valid. For example
  Pat's code wasn't the right thing to use for WDM models and here you
  can guard against that. }


\subsection{Full likelihood vs look-up table}

We will consider two type of users: the experts, who will like to access
the full likelihood, and the non-experts that will like to have a quick and 
easy access to the marginalized likelihood, i.e., where simulation details 
have been hidden away and astrophysical / nuisance parameters have already been 
marginalized over.

Every call of the \textit{full likelihood} involves the following steps:
\begin{itemize}
 \item 
\end{itemize}
we will need to to refer to the option




We expect to provide two different packages:
\begin{itemize}
 \item \textit{Slow likelihood} will return the un-marginalized likelihood
  for a given model, including both cosmological parameters and astrophysical
  or nuisance parameters (mean flux, thermal history, contaminants...).
  Evey 

\end{itemize}









The user will also specify:
\begin{itemize}
 \item Data products to use:
  SDSS-I from \cite{McDonald2006},
  BOSS from \cite{Palanque-Delabrouille2013},
  HIRES/UVES from \cite{Viel2013},
  XQSO-100 from \cite{Irsic2017},
  HIRES/UVES from \cite{Walther2018a}.
  We should also allow the user to specify what redshifts to use from each
  dataset, since they are independent, and probably also allow the user
  to specify the scales to be used in the fit for each dataset.
 \item Extra analysis settings, like whether to allow for running in the
  linear power, differences in the linear growth, or differences in the BAO
  wiggles.
\end{itemize}

\AFR{It is not clear to me whether at this point the user would also be able
to set other settings of the likelihood or not, like the way we treat
contamination by DLAs or metals, resolution or noise corrections, or the
parameterization of the temperature-density relation in the simulations.}
\as{This is implementational detail, no need to worry about this now.}

\AFR{I believe the answer is that each of these analysis choices would be
a different likelihood object, and then the user can decided whether to
use the likelihood object where the DLA marginalization used the formula
from \cite{McDonald2005b} or whether to use the one that used the formula
from \cite{Rogers2018a}.
Similarly, every time we want to use a different prior for the nuisance
parameters (say we want to include measurements of mean flux or temperature)
we would need to recompute the marginalization, and provide a new likelihood
look-up table.}

The output from the \textit{likelihood code} will be:
\begin{itemize}
 \item A value of (log-)likelihood for each dataset, possibly a value for 
  each redshift bin. 
  Most users will only care about this.
 \item For the experts, we would also output the best-fit theoretical 
  prediction for each dataset, for that particular cosmological model.
  We would also provide the values of the nuisance parameters that correspond
  to the best fit model (mean flux, temperature-density relation, metal or DLA
  contamination...).
 \item We could also provide a random sample of theory lines that are above a 
  likelihood threshold for that particular model, exploring different 
  thermal histories and other nuisance parameters.
\end{itemize}

\AFR{Actually, I don't think at this point you can get any of the above.
The user of the final likelihood code will only get a total value for the
likelihood for the total dataset.
Even though one would compute the un-marginalized likelihood for each redshift
and each dataset separately, at the step of marginalizing over nuisance 
parameters (see section \ref{sec:emu}) you would need to include them 
all at the same time.}
\as{No, no. You don't marginalise over nuissance parameters inside
  your likelihood code. Some nuissance parameters will have particular
  degeneracies with the big picture cosmological parameters, and in
  addition you never know how clever the user's sampler is. It is ok
  to ask for linear power, $f$, $H(z)$, $D_A(z)$ and a set of
  nuissance parameters, which users doesn't need to understand except
  for valid ranges. You can add a small wrapper that does this
  marginalisation internally for those who don't care.}
\pvm{In principle I more or less agree with \anze\ about the nuisance 
parameters, the core level of the code should be written treating all 
parameters symmetrically and this should be kept accessible... although,
if you were, e.g., importance sampling a Planck chain you would effectively
return to needing the code to marginalize over nuisance parameters. On the 
other hand, I remember long ago in Alexey Makarov days we ended up retreating
from this idea because the generic MCMC was painfully slowed by marginalization
over these nuisances when we knew how to make it fast internally... Ugh - 
I'm going to try not to read anything but emulator section since I think this
discussion of final public likelihood code is getting way ahead of yourself... 
\smiley\
I'd focus on ``what do I need to do to get an emulator capable of any kind of
fit to a LyaF data set?" ...  }

\AFR{Nooooooo! I need to understand how the final package will be used, 
before I can think of emulators and simulations!}

\AFR{Pat, in your 2005 paper, you had $\chi^2$ look-up tables, that were only
a function of amplitude and slope (and may be running). 
I always thougth that this type of look-up table would be the end product of 
the analysis, with a nice wrapper around it with a public interface to 
translate cosmology to our cryptic parameters in the lookup table.
Isn't it the case? 
You made the look-up table public, and that table had already marginalized
over all nuisance parameters, right?}
\pvm{Yes. This is getting at what I rambled about in comment later. While
\anze's desired option to control all parameters simultaneously in a cosmoMC
run should certainly *exist,* I don't think you should present this as, e.g., 
what anyone would run in their first use of your results. I do think you still
want to produce some form of ``de-forested'' linear power main result, that
will contain everything most people would want. You need it for 
``opening the black box" purposes, but also, it will be faster, and why force
people to mess with a bunch of forest nuisance parameters they don't care 
about when you know they can get the same results without it? The option to
run global chain with LyaF nuisances should exist for comparison, and so you
can see correlations, but not be the lead option... (this is sort of what I
was trying to indicate here saying ``core level of code", but I didn't really
think through how necessary I think the marginalized version really is...).
... I don't think this has to be very ``cryptic" even under the hood though...
I wouldn't call a $\chi^2$ table in $\Delta^2_L, \neff(z=3,k=0.009 s/km)$ 
cryptic... } 

\NEW{Perfect. Just for the record, in my original text (two days ago) I never
meant to talk about the CosmoMC option, I always had in mind to compress it
all to a look-up table with $\Delta^2_L, \neff(z=3,k=0.009 s/km)$ or 
equivalent (may be extending to $f_\star$ or running.}

\pvm{Ok, since it isn't actually perfectly trivial to define 
$\Delta^2_L, \neff(z=3,k=0.009 s/km)$ in practice, e.g., do you really want to
do an infinitesimal derivative at exactly this point or some kind of broader
thing, I think you will always want to control how that is done by asking 
for full P(k), and they should be happy to let you take care of it, so maybe it
could get a little cryptic, but this is not much related to the more physical
stuff related to emulator.} 
\NEW{Agreed.}
\as{I agree as well.}




\subsection{From cosmological model to likelihood parameters}

Under the hood, we will use more effective (and cryptic) parameters in our 
likelihood, to reduce the internal degeneracies between parameters.
There are many, many possibilities here, but we will start by discussing a 
possible setting. 

\subsubsection{Fiducial cosmological model}
 
We will choose a fiducial cosmological model, based on a recent Planck+BAO 
analysis \cite{Planck2015}, and use it to compute a fiducial linear power 
spectrum, $P_L^0(z,k)$, a fiducial Hubble expansion, $H^0(z)$, a fiducial 
growth rate $f^0(z)$...
All quantities with a superscript $^0$ will refer to the fiducial model.

\subsubsection{Linear power shape}

Since we have decided to use only models where the linear power spectrum 
can be factorized, we will describe its shape at the central redshift only, 
$z_\star=3$. 
In general we will use the subscript $_\star$ to refer to quantities that 
have been evaluated at $z_\star$, but in this sub-section we will ignore 
the redshift and assume that all power spectra are evaluated at $z_\star$.

We will compress the difference between the input power spectrum, $P_L(k)$, 
and the fiducial power spectrum, $P^0_L(k)$, into a handful of parameters. 
We start by fitting the fiducial power with a smooth function, 
$P^0_{nw}(k)$, using the \textit{no-wiggle} model from \cite{Eisenstein1998}.
We define the oscillatory (or \textit{wiggle}) component of the 
fiducial power, $W^0(k)$, as the ratio of these two powers:
\begin{equation}
 W^0(k) = \frac{P^0_L(k)}{P^0_{nw}(k)} ~.
\end{equation} 
\pvm{Are your boxes actually going to be big enough to worry about BAO? If 
so, I think you should *definitely* model that part by linear bias/more 
general PT... you can use that to pull out the feature essentially 
analytically. Never mind what I said about ``units of BAO"... this is surely 
the way to go since you don't want to be trying to interpolate between raw
sim power at that scale anyway...}
\NEW{No, in the main analysis I would not vary $r_d$ as a free parameter. 
But talking about this option might help others understand what we are doing, 
even though we then say: "we do not expect our results to be sensitive to
BAO, and therefore we fix $r_d$ to Planck value".} 

We will assume that the oscillatory component of the input model can be 
described with the oscillations in the fiducial model, shifted by the ratio 
of their sound horizons at the drag epoch ($r_d$):
\begin{equation}
 W(k) \sim W^0(\beta k)  
\qquad  
 \beta = \frac{r_d}{r^0_d} ~.
\end{equation}
We have decided to use $\beta$ and not $\alpha$, more common in BAO analyses,
because the latter includes a ratio of transformations from observable to 
commoving coordinates, and we do not need this at this point. 

Finally, we will model the differences in the smooth component with a smooth,
parameterized function:
\begin{equation}
 B(k) = \frac{P_{nw}(k)}{P_{nw}^0(k)} ~.
\end{equation}
There are different parameterizations possible, but for the rest of this 
discussion we will assume that we use three parameters: an amplitude, 
a slope, and a running of the slope, evaluated around $k_p = 1 \iMpc$. 

To summarize, we will describe the input linear power at $z_\star$ as:
\begin{equation} \label{eq:Pk_param}
 P_L(k) = B(k) ~ P_{nw}^0(k) ~ W^0(\beta k) ~.
\end{equation}

\subsubsection{Linear growth}

We will compute the difference of the input logarithmic growth rate with that 
in the fiducial cosmology, $\Delta f_\star = f(z_\star) - f^0(z_\star)$, 
evaluated at $z_\star=3$. 
We will approximate that the different growth rate at $z_\star$ is enough 
to compute the difference in linear growth at any redshift (within the range):
\begin{equation}\label{eq:growth}
 \frac{D(z)}{D^0(z)} = 1 + \Delta f_\star ~ \frac{\Delta z}{1 + z_\star} ~.
\end{equation} 
Note that in LCDM models, and at $2 < z < 5$, the differences in growth rate 
are typically less than 1\%, as shown in Figure \ref{fig:fz_Om}.

\subsubsection{Hubble expansion}

If we could observe the \lya\ power spectrum in comoving coordinates, that 
would be enough. 
However, we observe the power spectrum in observing coordinates, wavelengths
and angles, and a more natural choice is to use velocity units ($\kms$) for 
the clustering measurements. 
Indeed, all recent measurements of the 1D \lya\ power reported their results
in units of $\kms$, and we will assume the same in this discussion.

In general, we would need to use $H(z)$ from each model to compare 
measurements in $\kms$ with model preditions in $\Mpc$:
\begin{equation}
 q = \frac{1+z}{H(z)} ~ k = a_v k~,
\end{equation}
where we use $q$ to refer to wavenumbers in velocity units, and we have 
defined $a_v$ as the transformation from $\kms$ to $\Mpc$.
This would force us to add in the emulator some sort of Hubble 
parameter, either at $z=0$ ($h$) or at $z_\star=3$. 
However, as suggested by \cite{McDonald2005a}, it is possible to avoid this
burden if we describe our model (the linear power spectrum) already in 
units of $\kms$. \as{This confused the shit out of me until very
  recently. I vote for having power spectra in Mpc and also require
  $H(z_\star)$ and do this conversion internally -- it is trivial. }
\AFR{Yes, \anze, that was the entire point of the user interface. 
We ask the users to provide power in $\Mpc$, and we ask for $H(z)$, 
and we do the convertions under the hood.}

We claim that two models with different expansion histories $H(z)$, but the
same linear power in units of $\kms$, will have very similar \lya\ power
spectra, with small remaining differences being caused by astrophysical 
effects (different reionization history, different thermal history, different
mean flux...). 
And since we plan to marginalize over these to get the final cosmological 
constraints, we do not need to worry about these differences. 
For the rest of this discussion, we will assume that this is true.

\pvm{I agree with \anze\ that you should move to just using Mpc and asking for
$H(z)$, where, a clincher is in 3D you can't get
away from asking for at least $(H D_A)(z)$. The argument about km/s was always
that you should quote results as a linear power measurement in km/s at 
$z\sim 3$, *if you are going to quote results from LyaF $\sim$alone* -- the
goal there was to produce ``model independent" constraints that could be
propagated forward.}
\NEW{My plan was to translate the flux power into linear power, in $\kms$, 
and make that public as a look-up table, with a user interface in $\Mpc$ to 
make it easier for others to use. 
This is what I was trying to explain all along, mostly in section
\ref{sec:over}, but I guess I didn't do a good job \smiley.} \pvm{But if 
you're going to be using flux power directly in global chains fitting 
cosmological parameters, which seems to be what you're talking about here, 
that argument doesn't apply -- you have the only relevant cosmological model 
on hand point-by-point. }
\NEW{I never talked about doing this! Only \anze\ talked about it. 
I only talked about cosmological models as a wrapper, but the likelihood 
and the emulator would only hear about our compressed parameterization.} 
\pvm{This does raise the question though, how much you 
want to commit to that approach, i.e., to primarily present a likelihood code
that does direct fits to flux power measurements on the fly, vs. boiling 
things down to linear power measurement (which you would again obviously 
present in observable coordinates). Certainly internally you want to be able
to do both, and the code to do linear boil-down should probably wrap the other
one, but there is a question where to spend more time cleaning and advertising. 
Arguments in favor of boil-down to linear could be speed (in old days, like 
I said, we could do both, but fit to flux power really slowed down global
fits, while my $\chi^2$ table gave identical results), and just... it is nice
to be able to show the more-or-less LyaF-model-independent cosmological thing
you claim to have measured, not just present a big black likelihood box. 
It helps, e.g., estimate how these constraints will affect new cosmological  
models. In any case, you can figure out what to advertise later. It seems like
first goal should be to produce $\chi^2$ contours for, e.g.,  
$\Delta^2,\neff(z=3, k=0.009 s/km)$,  
defined as deviations from a central model (i.e.,
effectively variations of $A_s$ and $n_s$ with other parameters fixed), 
to compare to past... this is clearly going to be *most of* what matters.} 
\NEW{Yes, I agree with the above.}


How does this affect the discussion above?

Let us use $\tilde P(q)$ to refer to power spectra in units of velocity:
\begin{equation}
 \tilde P(q) = a_v^3 ~ P(k= q / a_v) ~.
\end{equation}

We can then redo the whole discussion above, but using power spectra in 
velocity units:
\begin{align} 
 \tilde P^0_L(q) & = (a^0_v)^3 ~ P^0_L(q / a^0_v)         \nonumber \\
   & = (a^0_v)^3 ~ P^0_{nw}(q / a^0_v) ~ W^0(q / a^0_v)      \nonumber \\
   & = \tilde P^0_{nw}(q) ~ \tilde W^0(q)  ~,
\end{align}
where we have also defined $\tilde W^0(q) = W^0(q / a^0_v0$.

We can now define a term for the ratio of the smooth power, in velocity
units:
\begin{align}
 \tilde B(q) & = \frac{\tilde P_{nw}(q)}{\tilde P^0_{nw}(q)}    \nonumber \\
  & = \left(\frac{a_v}{a_v^0}\right)^3 
      \frac{P_{nw}(q / a_v)}{P^0_{nw}(q / a_v^0)} ~,
\end{align}
and we can finally write 
\begin{align}
 \tilde P_L(q) & = \tilde P_{nw}(q) ~ W(q / a_v)                \nonumber \\
  & = \tilde B(q) ~ \tilde P^0_{nw}(q) ~ \tilde W^0(\alpha q) ~,
\end{align}
where we have defined $\alpha$ as the ratio of sound horizons in units of 
velocity: 
\begin{equation}
 \alpha = \beta \frac{a_v^0}{a_v} = \frac{r_d ~ H_\star}{r_d^0 ~ H^0_\star}~.
\end{equation}

The cosmological model in the likelihood will then be described by a 
set of parameters $\theta$ describing the linear power spectrum, including:
\begin{itemize}
 \item Ratio of sound horizons in units of $\kms$, $\alpha$, where the 
conversion from $\Mpc$ to $\kms$ is computed at $z_\star=3$. 
This is the inverse of the usual definition of BAO $\alpha_\parallel$.
 \item Approximately 3 parameters describing the ratio of the smooth
  linear power at $z_\star=3$, in units of $\kms$.
 \item Difference of growth rates at $z_\star=3$, $\Delta f_\star$. 
\end{itemize}

We will ask the likelihood code: for these set of parameters $\theta$,
what is the likelihood of getting the measured power, after marginalizing
over all nuisance parameters (including mean flux and thermal history)?
Or in math, what we want is:
\begin{equation} \label{eq:marg}
 L(\vd | \theta)
  = \int d\phi ~ \Pi(\phi) ~ L(\vd | \theta, \phi) ~,
\end{equation}
where $\vd$ is the measured flux power spectrum, $\phi$ are the nuisance
parameters, and $\Pi(\phi)$ are the priors on the nuisance parameters.
\as{Again, marginalisation is for the user to do. The code should
  output total $\chi^2$ at this point and optionally data-points and
  theory predictions at this point include full gory of nuissance
  parameters. If you look at eg \texttt{cosmomc} it internally
  marginalises already over some 15 Planck parameters, it can do 15
  more for us.}
\AFR{Ok, this is the type of discussion I wanted to have. 
Is this really how it was done in SDSS-I? What was the look-up table for, then?
To make plots?}
\pvm{I guess I should further to un-agree with \anze... 
If you wanted to assume everyone
will be using cosmoMC so you could just make a module for that and broadcast
it, I guess it would be ok (but I think still a lot of people would consider
your LyaF parameters to be too literally a nuisance... and be less tolerant 
of it than of Planck nuisance), but I don't think you want to think this way.
I don't think one MCMC code will dominate, so you need to assume people are
going to be grafting your likelihood code into various things themselves, 
so you want to make sure to have a very easy option. }
\NEW{Agreed. We should have an easy version for all to use, and a difficult
one for testing and for the experts to use.}

This likelihood will have been evaluated at a grid of points in $\theta$,
and it will be stored as a look up table.
Evaluating the likelihood, once the lookup table has been computed,
should then be trivial and extremely fast.



\section{The emulator} \label{sec:emu}

For a given combination of linear power parameters $\theta$, i.e., for
each point in the lookup table, we want to compute the integral in
equation \ref{eq:marg}.
To do this, we need to specify priors $\Pi(\phi)$, but more importantly
we need to compute the likelihood $L(\vd | \theta, \phi)$.

We need, therefore, to make predictions for the 1D flux power spectrum
at a certain set of points, $P_{1D}(z,q)$, in velocity units, given a model
defined by ($\theta$,$\phi$).
We discuss the (nuisance) astrophysical parameters $\phi$ in more detail
later on, but for now we will assume that there are only two parameters:
an overall normalization of the logarithm of the mean flux $\ln{\bar F_0}$,
multiplying some fiducial redshift evolution, and an overall normalization
of the filtering length $k_{F 0}$ (associated to the smoothing of the gas),
also multiplying a fiducial redshift evolution. 


\subsection{Emulating a particular redshift}

For each redshift $z_i$, we compute the corresponding value of the mean flux
($\bar F_i$) and filtering length ($k_{Fi}$).

Using the linear power parameters ($\theta$), and the fiducial
power spectrum (in velocity units), we are able to compute the expected
linear power for this particular model, in velocity units, at this redshift,
$\tilde P_i(q)$.

\AFR{This is one of the pieces that is still not crystal clear in my head.
Would we just take $\tilde P_L(z_\star,q)$, the linear power in velocity
units at the central redshift, and rescale it using the discussion around
equation \ref{eq:growth}?
If we did that, we would be missing the fact that the transformation between
comoving and velocity separations $a_v$ changes with redshift.
I guess one could use the fiducial model to compute this difference?
The alternative would be to have a 6th cosmological parameter describing
the difference in the change in the Hubble expansion around $z_\star$?}
\as{If DE is truly negligible for sensible models, then just don't
  worry about it and assumed EdS. If it makes small corrections, then
  the best course of action would be to also specify $d
  H^2/da=-3\Omega_m/a^4$ at $z=z_\star$.}
\pvm{I think there is a key thing you (Andreu) should add to your thinking 
about these things: don't focus on the parameters you put in when running the
simulation, focus on the effective parameters you *achieve* for each redshift 
output, i.e., the numbers you can associate most directly with the flux power 
spectrum produced from that redshift output. 
E.g., there is a linear power spectrum associated with each redshift
output, which you can easily compute using CLASS -- you don't really care
where it came from in terms or evolution from higher z, or, if you do, it is 
only as a very subdominant correction.}
\NEW{Yes, I got that. That's what I was trying to say here...}
\pvm{Even if you decided you needed to 
track differences in evolution for fixed output-time linear power, you would
probably want to do that by extracting $dP_{lin}/dz(z_{output})$ 
from CLASS, i.e., 
keep everything you associate with an output local in $z$. Going on, there is
an $F(z_{out})$, there is a $T(z_{out})$, there is a $k_F(z_{out})$ -- 
there is no
need to talk about a ``fiducial z" at all at this level. You only need to think
about that at a higher level of fitting, when, e.g., you want to produce 
$\chi^2$ contours in $\Delta^2_L(z=3), \neff(z=3)$ plane (fixing linear power
at other z assuming some model), or you want to enforce physically reasonable
temperature evolution connecting $T(z)$, $k_F(z)$. This isn't an entirely 
non-trivial attitude. E.g., if you didn't think you could summarize pressure
by a $k_F(z)$ you could calculate for each output, maybe you'd want to 
associate a full temperature history with each output instead of only 
writing down local-in-z quantities, and then you might want to parameterize
that thermal history somehow, but I would worry about that only when pushed to
it (and probably it would always be better to invent some local-in-z quantity
you could compute to capture the physical effect you were missing). 
To put it another way: you want to separate your picture into things you can
calculate about the conditions in the Universe at a given z without sims 
(including if necessary derivatives) -- you want to take advantage of these
kinds of things
as much as possible -- and then a simulation mapping of those
things into non-linear power (in more or less arbitrary units, followed by 
observation, applying the necessary units -- this part I think is easy for
everyone to agree on). }
\NEW{Yes, that was my plan.}
\pvm{This is an opportunity to say something I've been thinking about all this 
including neutrino, etc., sim testing: by *far* the most efficient way to 
test whether an idea you have for simplifying the emulator parameterization is
good enough is
to just do the simple version and then see if it works in the case you think it
might not. Probably it will work, and if not you haven't lost anything since
you should just need to expand parameter space a little and add some sims to 
probe the new effect, still using what you have done (assuming it was sensible
and the addition is more or less perturbative).  }

\AFR{Pat, that is what I was trying to write... See discussion below, about
computing some quantities at the particular redshift, and then completely 
dropping the redshift altogether.}

\AFR{But I don't think I've got an answer to my question. 
We define a model, by choosing $\theta$ (linear power at $z_\star=3$, 
in $\kms$, and may be $f_\star$) and by choosing $\phi$ ($\bar F$, $k_F$...);
We then ask the emulator to give us the prediction for flux power spectrum
at $z=4$, and the emulator needs to figure out how to translate this model
($\theta$,$\phi$) to the parameters describing the simulations outputs, to 
figure out which one to use (imagine we have infinit simulations);
For the IGM parameters it is easy, we have a way to use $\phi$ parameters
and the fiducial evolution, and turn that into a prediction for 
$\bar F_i$ and $k_{F~i}$;
However, we also need a way to translate the $\theta$ parameters, and the 
fiducial cosmological model, to a linear power at $z_i=4$, in $\kms$. 
How do we do this? 
If we had still access to the full $P(z,k)$ and $H(z)$ that entered the 
user interface, that would be trivial. 
But we threw that information away because we claimed that the only thing 
that matters (the only parameters in our final likelihood) were there 
$\theta$ parameters.
So we need to be able to reconstruct any linear power from these, so that 
then we can look at the simulations and try to look for a "snapshot" 
(more precisely, one of the multiple reproccessed snapshots) that had 
this particular power spectrum.
I'm not sure this is any clearer... 
}
\pvm{*For the emulator*, who says you need to ``define a model" by choosing 
power at $z_\star=3$? Where by ``emulator" I mean this thing that takes 
some kind of relatively easy to compute quantities and produces what it thinks
would be simulation results given them. Maybe it is easiest to think of it 
by sort of back-propagation: you have a 1D flux power spectrum measurement at 
$z=4$, you need a prediction for it, what does your emulator need to know to 
predict it? I'd say at first approximation it needs to know $P(k,z=4)$, in 
km/s. So the input to the emulator is $P(k,z=4)$ in km/s, period, end of 
story for emulator -- it has a hard enough job doing this well, it doesn't
need to worry about where this $P(k in km/s,z=4)$ came from. }
\NEW{Yes, I agree. I realize now that I was using the word \textit{emulator}
in the wrong way, including what you call bellow "code to make a $\chi^2$ 
table".}
\pvm{If you're asking
like ``how would I use this emulator to make a $\chi^2$ table of final results 
for $\Delta^2,\neff(z=3,k=0.009 s/km)$", worrying about broader z (and k) 
dependence,
I think you just pick the current best cosmological model as fiducial, and
define $\Delta^2,\neff(z=3,k=0.009 s/km)$ effectively as variations of 
$A_s$ and $n_s$ -- this gives you your $P(k in km/s,z=4)$ to feed the 
emulator, but it is really a completely separate thing from the emulator. }
\NEW{Just to be sure. $\theta$ parameters describe the different shape of the 
linear power, in $\kms$, at $z_\star=3$. 
I can try to use these, and $H^0(z_\star)$ from the fiducial model, to compute 
the equivalent linear power in comoving coordinates.
Then I can translate that to a different redshift using the linear growth
of the fiducial model (may be corrected by difference in $f_\star$), and 
then use the Hubble parameter of the fiducial model at the redshift, 
$H^0(z_i=4)$ to compute the final power we needed to talk to the simulations.
Correct? The only thing this could break is if the redshift evolution of 
$H(z)$ and $H^0(z)$ were very different, but that should not be the case for
most models, and if it was we could add an extra parameter to take care of 
this. Did I understand it?}
\pvm{Of course, maybe you want to fit for growth deviations, and this gives you 
a different way of getting $P(k in km/s,z=4)$, or maybe you are doing a big
global MCMC chain... the emulator who's job is to predict 
1D flux power at z=4 doesn't want to know what you are doing globally, it 
just wants to know $P(k in km/s,z=4)$... (I've been writing 
$P(k in km/s,z=4)$ because I carefully wrote 1D power and it is shorter than
writing ``$P(k in Mpc,z=4)$ and $H(z=4)$", which I think we agree is probably
how things should really go for pedagogical reasons, and 
add $D_A(z=4)$ for 3D.) }
\NEW{Ok, I think we are getting closer. I think part of the confusion was 
my poor use of the word \textit{emulator}, and the other part of the confusion
is that I always wanted to focus on the "look-up table" version, and not on 
the cosmomc version.}



At this point, we can completely forget about the redshift $z_i$.

Instead, we will go to our simulation database, and ask:
is there any simulated flux power spectra that was computed from a snapshot
with similar linear power (in velocity units) $\tilde P_i(q)$, similar
mean flux $\bar F_i$ and similar filtering length $k_{Fi}$? \as{And
  similar $f$ for velocity effects? Again, only matters if non-EdS matters.}
\pvm{Remember that much of non-EdS effects can still be accounted for by just
linear theory. } \as{But it changes growth, which in turn changes
velocity smoothing. So yes, linear power spectrum but also most likely
its time derivative (or equivalently velocity power spectrum)} \pvm{Thinking of neutrinos, I think it is really best to get away
from talking about $f(z)$, which is not well-defined when it is really 
$f(z,k)$. 
Remember that you can easily compute from CLASS the velocity power spectrum as
well as density, which gives you your leading order handle on changes in 
evolution... (arguably if you had to choose you'd probably want this instead
of density power for LyaF, but you don't have to choose...)}
\AFR{Yes, we could add linear velocity power instead of $f(z)$, and compute
from there any parameter we want to use internally.}

If we had an extremely large number of simulations, we could just setup a
metric to find the closest snapshot, and directly read the flux power
from the snapshot.
Since we will have a sparse sampling of the parameter space, we will need
to do some interpolation between them. \as{This is a good way to think
  about this, yes.}
This interpolation is precisely the role of the \textit{emulator}.

Note that the internal metric used for the interpolation does not need to 
use the same parameters $\theta$ describing the linear power.



\section{Simulations}
\label{sec:sims}

Describe here the simulations, include the initial conditions code (GenIC), 
the code to evolve the fields (MP-Gadget), the different boxes used, 
and the code to extract skewers (fake\_spectra).

Discuss also here the optical depth rescaling, different transfer functions
(if any), and thermal history effects (that will be mostly ignored in this
paper).


\subsection{Rescaling of the optical depth}

From each simulation we will get a set of snapshots, outputs at different 
redshifts. 

From each snapshot, we will extract \lya\ skewers, and use these to compute 
their power spectra. 
We will repeat this exercise for different rescalings of the optical depth, 
i.e., we will multiply the optical depth in all cells by a constant factor
in the range $0.8 < A_\tau < 1.2$ (approximately), and for each value of 
$A_\tau$ we will compute the transmitted flux fraction $F$, its mean value
(mean flux), and the power spectra of their fluctuations $\delta_F$. 

Therefore, from each snapshot we will get a set of power spectra for different
values of $A_\tau$.
We can label the different power spectra by their associated value of $A_\tau$,
or we can label them by their resulting value of the mean flux $\bar F$. 

\cite{Lukic2015} showed that the rescaling might introduce biases in the 
1D power spectrum for values of $A_\tau$ very different than one. 
However, their test compared two simulations with different thermal history
and different pressure, so it is difficult to tell whether the bias came 
from the rescaling or from the different IGM physics.

\AFR{It would be great to repeat this exercise in two simulations that 
have very similar thermal history but different mean flux, I will ask 
Jose Onorbe for help (he is visiting UCL soon). 
It is also possible that the test is clearer if we look at the 3D power,
where pressure only affects the high-k limit, and the scale independent
linera bias.}


\subsection{Rescaling of the temperature}

The Temperature-Density Relation (TDR) in the \lya\ forest can be reasonably
well described by a power law, 
\begin{equation}
 T(\rho) = T_0 \left(\frac{\rho}{\rho_0}\right)^{\gamma-1} ~,
\end{equation}
with a typical values for $\gamma$ between 1 and 1.6. 
If we use $\rho_0 = \bar \rho$, $T_0$ varies between 10,000 and 20,000K
\cite{Lukic2015}.

We can change the thermal history of the simulations by running the same box
with different \textit{TREECOOL} files, that contain the redshift evolution
of different heating and ionizing rates.

For each snapshot, we can fit their values of $T_0$ and $\gamma$, and use 
these to label the snapshot, instead of using the name of the TREECOOL, 
or the parameters that we have used to modify a given file 
(MP-Gadget can implement the recipes in \cite{Bolton2008} to modify TRECOOL
files by setting the parameters \textit{HeatAmplitude} and \textit{HealSlope}).

Just like we did with the optical depth, we can rescale the temperatures in 
post-processing. 
We could do this at two different levels:
\begin{itemize}
 \item Thermal broadening: when extracting the skewers from the boxes, 
  the last step is to compute the redshift-space-distorted optical depth. 
  To do that, the temperature at each cell is used to decide the local 
  smoothing that we need to apply, what is known as the thermal broadening. 
  It is trivial to take the same optical depth skewer, and convolve it for
  different temperature rescalings.
 \item Recombination rates: the temperature also sets the recombination rate,
  $\alpha(T) \sim T^{-0.7}$. 
  It would also be possible to change the recombination rate at each cell, 
  propagate this into a change in the neutral fraction (proportional to the 
  recombination rate), and finally to the optical depth (proportional to the 
  neutral fraction). 
\end{itemize}

Note that, even though this would capture most of the effects of having a 
different temperature, we would miss the effect that different temperature
have in the pressure smoothing.
However, we will include a parameter to study the dependence on the pressure
smoothing, the filtering length $k_F$, and this should be able to capture
the differences.


\subsection{Adding pressure smoothing}

The small scale structure is suppressed on very small scales because of the
pressure in the gas. 
As described in \cite{Hui1997,Gnedin1998}, the smoothing can be described
by a characteristic scale, $k_F)(z)$, the \textit{filtering scale}, that is
an integral version of the Jeans length that depends also on the temperature
in the past.

It would be interesting to see if we can add smoothing in postprocessing, 
effectively lowering the value of $k_F$ in the simulation by hand. 
Of course, it would not be possible to reduce the smoothing, and to do that 
one would need to run a simulation with an earlier redshift of reionization.

\AFR{We could also ask Jose Onorbe for help to setup this type of test. 
We would also need to discuss the best way to measure $k_F$ in the snapshots:
fit a Gaussian kernel in the power spectrum of $F_{\rm real}$, where no 
redshift-space distortions have been included? I believe that is what is 
used in all papers by Hennawi / Lukic / Onorbe.}


\subsection{Labelling the snapshots}

In the linear regime, and for a single specie, the growth of structure is 
scale independent, and it can be described by the growth factor $D(z)$.
%\begin{equation}
% P_L(z,k) = \frac{D^2(z)}{D^2(z_0)} ~ P_L(z_0,k) ~.
%\end{equation}

If the \lya\ power spectra depended only on the linear power spectrum, this 
would suggest that there would be a complete degeneracy between changing 
the overall amplitude of the linear power spectrum and changing the redshift
at which we ouput the snapshot. 
Therefore, we could use the amplitude of the linear power at a given snapshot
to label it, and use the different snapshots of the same simulation to study 
models with different amplitudes of the linear power. 

In section \ref{sec:dens} we will check whether this is still true at the 
level of the non-linear power spectrum, and in section \ref{sec:lya} we 
will look at this degeneracy in the \lya\ power spectrum. 

\AFR{How do we measure the linear power in the snapshot? 
I could see three options (in order of my preference): 
predict it using the power measured in the initial conditions, and the relative
growth as computed from CAMB/CLASS;
measure the density power in the snapshot, and fit the growth factor from 
the low-k part;
run a very cheap simulation without hydro and a very low value of $A_s$, to 
compute the actual linear power in the simulation (that might sadly differ 
from the predicted by CAMB/CLASS because of issues in the linear growht).}

To sum up, each snapshot will be used to generate multiple simulated 
fields, and we can label each of them by their values of: 
mean flux (1 parameter) $\bar F$,
linear power (3 parameters) $P_L$,
TDR (2 parameters) $T_0$ and $\gamma$, 
filtering scale $k_F$.

Redshift will NOT be a label describing the simulated field, and neither
will be the TREECOOL file or the redshift of reionization.
Since we will define the linear power in units of $\kms$, we will not 
need to include the Hubble parameter at the box as a label.
We will not care about any other cosmological parameter in the box
either.


\section{Fisher forecasts} \label{sec:fore}

\AFR{I will add here some discussion on how one would use the same set of
simulations to compute Fisher forecasts for the 1D power, the 3D power and
different type of bispectrums.}



\section*{Acknowledgements}
AFR acknowledges support by an STFC Ernest Rutherford Fellowship, grant reference ST/N003853/1.
%AS acknowledges hospitality of the University College London.
This work was partially enabled by funding from the UCL Cosmoparticle
Initiative.

\bibliography{main}
\bibliographystyle{JHEP}

\appendix
\section{Derivation of equations in the main text} \label{app:eq}

I write below the equations deriving equation \ref{eq:growth}.

For the input cosmology, we define:
\begin{equation}
 P(z,k) = P_\star(k) \left[ \frac{D(z)}{D_\star} \right]^2 ~,
\end{equation}
where $D(z)$ is the growth factor and $_\star$ means that the quantity is 
evaluated at $z_\star=3$. 

The evolution of the growth factor is quite similar to Einstein-de Sitter 
(EdS), i.e., $D(z) \propto a(z)$, and we define the deviation from that growth 
as follows:
\begin{equation}
 \frac{D(z)}{D_\star} = \frac{a(z)}{a_\star} ~ \eta(z) ~,
\end{equation}
with $\eta_\star=1$ by definition and $\eta(z)=1$ in an EdS universe.

We can then do a Taylor expansion of $\eta(z)$ around $z_\star$:
\begin{align}
 \eta(z_\star+\Delta z) 
  & = 1 + \frac{\partial \eta}{\partial z} 
      \Bigr\rvert_{z_\star} \Delta z                    \nonumber \\
  & = 1 - a_\star^2~\frac{\partial \eta}{\partial a} 
      \Bigr\rvert_{z_\star} \Delta z                    \nonumber \\
  & = 1 + \left( 1 - f_\star \right) \frac{\Delta z}{1+z_\star} ~,
\end{align}
where we have used 
\begin{equation}
 \frac{\partial \eta}{\partial a} \Bigr\rvert_{z_\star} 
  = \frac{a_\star}{D_\star} \frac{\partial D}{\partial a} \Bigr\rvert_{z_\star}
  = \frac{1}{a_\star} \left( f_\star - 1 \right) ~.
\end{equation}

\begin{figure}[h]
 \begin{center}
  \includegraphics{fz_omega_m}
 \end{center}
 \caption{Logarithmic growth rate, $f(z)$, for different cosmologies.
  Solid lines show the ratio with respect to fiducial ($\Omega_m=0.3$), 
  and dashed lines the value at $z_\star=3$, assumed to be constant in this 
  paper.}
 \label{fig:fz_Om}
\end{figure}


\newpage
\section{Reconstructing the linear power from the parameters} \label{app:P_L}

In section \ref{sec:like} we discussed how we compress the cosmological
information provided by the user to a handful of $\theta$ parameters
describing the linear power spectrum over the relevant redshift and scales.
This reduced set of parameters $\theta$ will be the ones used in the
compressed (or marginalized) likelihood.

In order to compute the marginalized likelihood we need to use the default
cosmological model ($P^0(z,k)$, $P_\theta^0(z,k)$ and $H(z)$) and the set
of $\theta$ parameters to compute the corresponding linear power at a
particular redshift $z_i$ and in velocity units, $\tilde P_i(q)$, as well
as the growth rate in that redshift $f_i$.
In this section I will describe in detail how we do this.

\begin{itemize}
 \item Using the linear power in the fiducial model $P_\star^0(k)$ and the
  Hubble parameter $H^0_\star$, both evaluated at $z_\star=3$, we compute
  the linear power spectrum in velocity units for the fiducial model,
  $\tilde P^0_\star(q)$.
 \item Using the shape parameters $\beta$, we compute the linear power
  spectrum in velocity units for the input model, $\tilde P_\star(q)$.
 \item Using the Hubble parameter in the fiducial cosmology $H^0_\star$
  we convert this to a linear power spectrum in comoving coordinates,
  $P_\star(k)$.
 \item Using the linear growth in the fiducial cosmology, $f_0(z)$, we
  translate that to the desired redshift, $P(z_i,k)$.
  If we had $\Delta f_\star$ as a free parameter, we would use that to
  slightly modify this redshift evolution.
 \item We finally use the Hubble expansion in the fiducial cosmology,
  $H^0(z_i)$, to convert the linear power in velocity units at the desired
  redshift, $\tilde P(z_i,q)$.
  If we had an extra parameter describing the different expansion history,
  we would use it here to slightly modified this last step.
\end{itemize}

Another way to look at it, is that we will take the linear power at 
$z_\star=3$, in velocity units, and try to find a modification of the
primordial power in the fiducial model that matches the power.
For instance, by trying to find different values of ($A_s$,$n_s$ and 
$\alpha_S$) with which the fiducial model can be modified to match the
input power. 
We would then use call CAMB/CLASS with the fiducial model cosmology, but
this modified primordial power, to make predictions for the linear power
at any redshift.
This, of course, is for cases where we ignore differences in $f(z)$ or
$H(z)$, but these could easily be introduced as well.

\AFR{I actually prefer the itemized description, but it is a matter of taste.}

\afhid{But I don't think I've got an answer to my question. 
We define a model, by choosing $\theta$ (linear power at $z_\star=3$, 
in $\kms$, and may be $f_\star$) and by choosing $\phi$ ($\bar F$, $k_F$...);
We then ask the emulator to give us the prediction for flux power spectrum
at $z=4$, and the emulator needs to figure out how to translate this model
($\theta$,$\phi$) to the parameters describing the simulations outputs, to 
figure out which one to use (imagine we have infinit simulations);
For the IGM parameters it is easy, we have a way to use $\phi$ parameters
and the fiducial evolution, and turn that into a prediction for 
$\bar F_i$ and $k_{F~i}$;
However, we also need a way to translate the $\theta$ parameters, and the 
fiducial cosmological model, to a linear power at $z_i=4$, in $\kms$. 
How do we do this? 
If we had still access to the full $P(z,k)$ and $H(z)$ that entered the 
user interface, that would be trivial. 
But we threw that information away because we claimed that the only thing 
that matters (the only parameters in our final likelihood) were there 
$\theta$ parameters.
So we need to be able to reconstruct any linear power from these, so that 
then we can look at the simulations and try to look for a "snapshot" 
(more precisely, one of the multiple reproccessed snapshots) that had 
this particular power spectrum.
I'm not sure this is any clearer... 
}
\pvmhid{*For the emulator*, who says you need to ``define a model" by choosing 
power at $z_\star=3$? Where by ``emulator" I mean this thing that takes 
some kind of relatively easy to compute quantities and produces what it thinks
would be simulation results given them. Maybe it is easiest to think of it 
by sort of back-propagation: you have a 1D flux power spectrum measurement at 
$z=4$, you need a prediction for it, what does your emulator need to know to 
predict it? I'd say at first approximation it needs to know $P(k,z=4)$, in 
km/s. So the input to the emulator is $P(k,z=4)$ in km/s, period, end of 
story for emulator -- it has a hard enough job doing this well, it doesn't
need to worry about where this $P(k in km/s,z=4)$ came from. }
\afhid{Yes, I agree. I realize now that I was using the word \textit{emulator}
in the wrong way, including what you call bellow "code to make a $\chi^2$ 
table".}
\pvmhid{If you're asking
like ``how would I use this emulator to make a $\chi^2$ table of final results 
for $\Delta^2,\neff(z=3,k=0.009 s/km)$", worrying about broader z (and k) 
dependence,
I think you just pick the current best cosmological model as fiducial, and
define $\Delta^2,\neff(z=3,k=0.009 s/km)$ effectively as variations of 
$A_s$ and $n_s$ -- this gives you your $P(k in km/s,z=4)$ to feed the 
emulator, but it is really a completely separate thing from the emulator. }
\afhid{Just to be sure. $\theta$ parameters describe the different shape of the 
linear power, in $\kms$, at $z_\star=3$. 
I can try to use these, and $H^0(z_\star)$ from the fiducial model, to compute 
the equivalent linear power in comoving coordinates.
Then I can translate that to a different redshift using the linear growth
of the fiducial model (may be corrected by difference in $f_\star$), and 
then use the Hubble parameter of the fiducial model at the redshift, 
$H^0(z_i=4)$ to compute the final power we needed to talk to the simulations.
Correct? The only thing this could break is if the redshift evolution of 
$H(z)$ and $H^0(z)$ were very different, but that should not be the case for
most models, and if it was we could add an extra parameter to take care of 
this. Did I understand it?}
\pvmhid{In practice, I think you want to use $\Delta^2,\neff(z=3,k=0.009 s/km)$
to fix $A_s$ and $n_s$ in your fiducial model (i.e., the rest of it is fixed,
but they are varied), then you compute linear power at whatever redshift you
want directly from CLASS. I'm sure this is equivalent to what you are 
saying about scalings, and I wouldn't even object if you actually did those
scalings, but you should understand that they are a shortcut for this 
$A_s, n_s$ setting $\rightarrow$ CLASS run, not anything else. Of course, 
if $H(z)$ or something deviates enough from your fiducial model this 
single-amp-slope
compression won't be good enough, but that is a different issue (an extended
table would add running of spectral index and arbitrary power law modification
of growth factor, allowing you to do fits testing consistency of those 
things... and I should say, with improved precision maybe deviations within 
realistic models will be big enough to matter... I'm not saying they won't,
just that you should have a baseline idea and then think about whether you
need to expand).  }
\pvmhid{Of course, maybe you want to fit for growth deviations, and this gives
you a different way of getting $P(k in km/s,z=4)$, or maybe you are doing a big
global MCMC chain... the emulator who's job is to predict 
1D flux power at z=4 doesn't want to know what you are doing globally, it 
just wants to know $P(k in km/s,z=4)$... (I've been writing 
$P(k in km/s,z=4)$ because I carefully wrote 1D power and it is shorter than
writing ``$P(k in Mpc,z=4)$ and $H(z=4)$", which I think we agree is probably
how things should really go for pedagogical reasons, and 
add $D_A(z=4)$ for 3D.) }
\afhid{Ok, I think we are getting closer. I think part of the confusion was 
my poor use of the word \textit{emulator}, and the other part of the confusion
is that I always wanted to focus on the "look-up table" version, and not on 
the cosmomc version.}



\end{document}
